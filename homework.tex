\documentclass[8pt]{amsart}
\usepackage{amsmath, amssymb}
\usepackage{amsfonts}
\usepackage{mathrsfs}
\usepackage[arrow,matrix,curve,cmtip,ps]{xy}
\usepackage{paralist}
\usepackage[left=1in,top=1in,right=1in,bottom=1in]{geometry}
\usepackage{amsthm}

\allowdisplaybreaks

\theoremstyle{plain}% default

\theoremstyle{definition}
\newtheorem{theorem}{Theorem}[section]
\newtheorem{lemma}{Lemma}[section]
\newtheorem*{proposition}{Proposition}
\newtheorem*{corollary}{Corollary}
\newtheorem*{KL}{Klein’s Lemma}

\newtheorem*{definition}{Definition}
\newtheorem{conjecture}{Conjecture}[section]
\newtheorem{example}{Example}[section]
\newtheorem{exercise}{Exercise}[section]
\newtheorem*{notation}{Notation}
\newtheorem*{remark}{Remark}

\theoremstyle{remark}
\newtheorem*{note}{Note}
\newtheorem{case}{Case}

%this has equations numbered within sections 1.1,1.2, ... 2.1,...
\numberwithin{equation}{section}

\makeatletter
\newenvironment{solution}
               {\let\oldqedsymbol=\qedsymbol%
                \def\@addpunct##1{}%
                \renewcommand{\qedsymbol}{$\blacktriangleleft$}%
                \begin{proof}[\itshape Solution.]}%
               {\end{proof}%
                \renewcommand{\qedsymbol}{\oldqedsymbol}}
\makeatother

\def\upint{\mathchoice%
    {\mkern13mu\overline{\vphantom{\intop}\mkern7mu}\mkern-20mu}%
    {\mkern7mu\overline{\vphantom{\intop}\mkern7mu}\mkern-14mu}%
    {\mkern7mu\overline{\vphantom{\intop}\mkern7mu}\mkern-14mu}%
    {\mkern7mu\overline{\vphantom{\intop}\mkern7mu}\mkern-14mu}%
  \int}
\def\lowint{\mkern3mu\underline{\vphantom{\intop}\mkern7mu}\mkern-10mu\int}


%-------------------------------------------
%       Begin Local Macros
%-------------------------------------------

\newcommand{\Z}{\mathbb{Z}}
\newcommand{\N}{\mathbb{N}}
\newcommand{\R}{\mathbb{R}}
\newcommand{\C}{\mathbb{C}}
\newcommand{\T}{\mathbb{T}}
\newcommand{\D}{\displaystyle}
\newcommand{\im}{\operatorname{im}}
\newcommand{\coker}{\operatorname{coker}}
\newcommand{\ind}{\operatorname{ind}}
\newcommand{\rank}{\operatorname{rank}}
\newcommand\mc[1]{\marginpar{\sloppy\protect\footnotesize #1}}

%-------------------------------------------
%       End Local Macros
%-------------------------------------------
\begin{document}
\title[MATH 697]{Introduction to Category Theory}

% \author{Robert Cardona}

\author{
	Robert Cardona %\textit{mrrobertcardona@gmail}
	\and
	Massy Khoshbin %\textit{massy255@gmail.com}
	\and
	Siavash Mortezavi %\textit{siavash.mortezavi@gmail.com}
}


\address{Department of Mathematics \\ California State University Long Beach}
\email{mrrobertcardona@gmail.com \and massy255@gmail \and siavash.mortezavi@gmail}

\date{\today}

\begin{abstract}
Module Theory.
\end{abstract}

\maketitle

%%%%%%%%%%%%%%%%%%%%%%%%%%%%%%%%%%%%%%%%%%%%%%%%
\setcounter{section}{-1}
\section{MATH 697 Homework Zero}
%%%%%%%%%%%%%%%%%%%%%%%%%%%%%%%%%%%%%%%%%%%%%%%%

\begin{exercise}
Prove Theorem 4 (Isomorphism Theorems):
	\begin{enumerate}
		\item (\textit{The First Isomorphism Theorem for Modules}) Let $M, N$ be $R$-modules and let $\varphi : M \to N$ be an $R$-modules homomorphism. Then $\ker \varphi$ is a submodule of $M$ and $M/\ker \varphi \cong \varphi(M)$.
			\begin{proof}
				Let $M, N$ be $R$-modules and let $\varphi : M \to N$ be an $R$-modules homomorphism. Then by definition $\varphi(x + y) = \varphi(x) + \varphi(y)$ and $\varphi(rx) = r\varphi(x)$ for all $x, y \in M$, $r \in R$. We want to show that $\ker \varphi = \{m \in M : \varphi(m) = 0\}$ is a submodule. Observe that since $M$ is a module then $M$ is an abelian group by definition so there exists $0 \in M$ such that $m + 0 = m$ for all $m \in M$. In particular $\varphi(0) = \varphi(0 + 0) = \varphi(0) + \varphi(0)$ implying $\varphi(0) = 0$. Conclude that $0 \in \ker \varphi \neq \emptyset$. Now let $r \in R$, $x, y \in \ker \varphi$. Observe that $\varphi(x + ry) = \varphi(x) + \varphi(ry) = \varphi(x) + r\varphi(y) = 0 + r \cdot 0 = 0 + 0 = 0$. Hence $x + ry \in \ker \varphi$. Conclude by the submodule criterion that $\ker \varphi$ is in fact a submodule.\\

				Now define $\Phi : M/\ker \varphi \to \varphi(M)$ by $\Phi(m + \ker \varphi) = \varphi(m)$. We want to show that this mapping is a well-defined bijective homomorphism. We first show well-definedness. Suppose $m + \ker \varphi = m' + \ker \varphi$ it follows by property of cosets that $m - m' \in \ker \varphi$, in particular $\varphi(m - m') = \varphi(m) - \varphi(m') = 0$ and hence $\varphi(m) = \varphi(m')$. But since $\varphi(m) = \Phi(m + \ker \varphi)$ and $\varphi(m') = \Phi(m' + \ker\varphi)$ we have $\Phi(m + \ker\varphi) = \Phi(m' + \ker\varphi)$. Conclude that $\Phi$ is in fact well-defined.\\

				Suppose that $\Phi(m + \ker\varphi) = \Phi(m' + \ker\varphi)$. Then it follows that $\varphi(m) = \varphi(m')$ and so $\varphi(m - m') = 0$ and so $m - m' \in \ker\varphi$. By property of cosets it follows that $m + \ker\varphi = m' + \ker\varphi$ and hence $\Phi$ is injective.\\

				Let $n \in \varphi(M)$. Then by definition of image of $\varphi$ there exists $m \in M$ such that $n = \varphi(m)$. It is immediate that $m + \ker\varphi \in M/\ker\varphi$ and we can conclude that $\Phi$ is surjective.\\

				Now we must show that $\Phi$ is an $R$-module homomorphism. Let $x, y \in M/\ker\varphi$ where $x = m + \ker\varphi$ and $y = m' + \ker\varphi$ for some $m, m' \in M$ and let $r \in R$. Observe that
				\begin{align*}
					\Phi(x + y)&= \Phi(m + m' + \ker\varphi)\\
					&= \varphi(m + m')\\
					&= \varphi(m) + \varphi(m')\\
					&= \Phi(m + \ker\varphi) + \Phi(m' + \ker\varphi)\\
					&= \Phi(x) + \Phi(y)
				\end{align*}
				and
				\begin{align*}
					\Phi(rx) &= \Phi(r(m + \ker\varphi))\\
					&= \Phi(rm + \ker\varphi)\\ %absorbption? is it an ideal?
					&= \varphi(rm)\\
					&= r\varphi(m)\\
					&= r\Phi(m + \ker\varphi)\\
					&= r\Phi(x)
				\end{align*}
				Hence we have shown that $\Phi$ is a well-defined bijective homomorphism and thus we can conclude by definition of $R$-module isomorphism that $M/\ker\varphi \cong \varphi(M)$.
			\end{proof}
		\item (\textit{The Second Isomorphism Theorem}) Let $A, B$ be submodules of the $R$-module $M$. Then $(A + B)/B \cong A/(A \cap B)$.
			\begin{proof}
				Define $\varphi : A \to (A + B)/B$ by $\varphi(a) = a + B$. This mapping is clearly well-defined. We want to show that $\varphi$ is a homomorphism. Let $r \in R$, $a, a' \in A$ and observe that
					\begin{align*}
						\varphi(a + a') &= a + a' + B\\
						&= a + B + a' + B\\
						&= \varphi(a) + \varphi(a')
					\end{align*}
					and
					\begin{align*}
						\varphi(ra) = ra + B\\
						&= r(a + B)\\
						&= r\varphi(a)
					\end{align*}
					and so $\varphi$ is an $R$-module homomorphism by definition. Observe that $\ker \varphi = \{a \in A : \varphi(a) = 0\} = \{a \in A : a + B = 0\} = \{a \in A : a \in B\} = A \cap B$. Now let $x \in (A + B)/B$ then $x = a + b + B$ for some $a \in A$, $b \in B$. But observe that $a + b + B = a + B$ by absorbption. So $\varphi$ is immediately surjective. In particular we have $\varphi(A) = (A + B)/B$. Conclude by the First Isomorphism Theorem for Modules that $A/\ker \varphi = A/(A \cap B) \cong (A + B)/B = \varphi(A)$.
			\end{proof}
		\item (\textit{The Third Isomorphism Theorem}) Let $M$ be an $R$-module, and let $A$ and $B$ be submodules of $M$ with $A \subseteq B$. Then $(M/A)/(B/A) \cong M/B$.
		\item (\textit{The Fourth or Lattice Isomorphism Theorem}) Let $N$ be a submodule of the $R$-module $M$. There is a bijection between the submodules of $M$ which contain $N$ and the submodules of $M/N$. The correspondence is given by $A \leftrightarrow A/N$, for all $A \supseteq N$. The correspondence cummutes with the processes of taking sums and intersections (i.e., is a lattice isomorphism between the lattice of submodules of $M/N$ and the lattice of submodules of $M$ which contain $N$).
	\end{enumerate}
\end{exercise}



















\end{document}




