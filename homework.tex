\documentclass[8pt]{amsart}
\usepackage{amsmath, amssymb}
\usepackage{amsfonts}
\usepackage{mathrsfs}
\usepackage[arrow,matrix,curve,cmtip,ps]{xy}
\usepackage{paralist}
\usepackage[left=1in,top=1in,right=1in,bottom=1in]{geometry}
\usepackage{amsthm}

\allowdisplaybreaks

\theoremstyle{plain}% default

\theoremstyle{definition}
\newtheorem{theorem}{Theorem}[section]
\newtheorem{lemma}{Lemma}[section]
\newtheorem*{proposition}{Proposition}
\newtheorem*{corollary}{Corollary}
\newtheorem*{KL}{Klein’s Lemma}

\newtheorem*{definition}{Definition}
\newtheorem{conjecture}{Conjecture}[section]
\newtheorem{example}{Example}[section]
\newtheorem{exercise}{Exercise}[section]
\newtheorem*{notation}{Notation}
\newtheorem*{remark}{Remark}

\theoremstyle{remark}
\newtheorem*{note}{Note}
\newtheorem{case}{Case}

%this has equations numbered within sections 1.1,1.2, ... 2.1,...
\numberwithin{equation}{section}

\makeatletter
\newenvironment{solution}
               {\let\oldqedsymbol=\qedsymbol%
                \def\@addpunct##1{}%
                \renewcommand{\qedsymbol}{$\blacktriangleleft$}%
                \begin{proof}[\itshape Solution.]}%
               {\end{proof}%
                \renewcommand{\qedsymbol}{\oldqedsymbol}}
\makeatother

\def\upint{\mathchoice%
    {\mkern13mu\overline{\vphantom{\intop}\mkern7mu}\mkern-20mu}%
    {\mkern7mu\overline{\vphantom{\intop}\mkern7mu}\mkern-14mu}%
    {\mkern7mu\overline{\vphantom{\intop}\mkern7mu}\mkern-14mu}%
    {\mkern7mu\overline{\vphantom{\intop}\mkern7mu}\mkern-14mu}%
  \int}
\def\lowint{\mkern3mu\underline{\vphantom{\intop}\mkern7mu}\mkern-10mu\int}


%-------------------------------------------
%       Begin Local Macros
%-------------------------------------------

\newcommand{\Z}{\mathbb{Z}}
\newcommand{\N}{\mathbb{N}}
\newcommand{\R}{\mathbb{R}}
\newcommand{\C}{\mathbb{C}}
\newcommand{\T}{\mathbb{T}}
\newcommand{\D}{\displaystyle}
\newcommand{\im}{\operatorname{im}}
\newcommand{\coker}{\operatorname{coker}}
\newcommand{\ind}{\operatorname{ind}}
\newcommand{\rank}{\operatorname{rank}}
\newcommand\mc[1]{\marginpar{\sloppy\protect\footnotesize #1}}

%-------------------------------------------
%       End Local Macros
%-------------------------------------------
\begin{document}
\title[MATH 697]{Introduction to Category Theory}

% \author{Robert Cardona}

\author{
	Robert Cardona %\textit{mrrobertcardona@gmail}
	\and
	Massy Khoshbin %\textit{massy255@gmail.com}
	\and
	Siavash Mortezavi %\textit{siavash.mortezavi@gmail.com}
}


\address{Department of Mathematics \\ California State University Long Beach}
\email{mrrobertcardona@gmail.com \and massy255@gmail \and siavash.mortezavi@gmail}

\date{\today}

\begin{abstract}
Category Theory: Remain calm and carry on when all the mathematics you've ever known and loved gets abstracted away into dots and arrows.
\end{abstract}

\maketitle

%%%%%%%%%%%%%%%%%%%%%%%%%%%%%%%%%%%%%%%%%%%%%%%%
\setcounter{section}{-1}
\section{MATH 697 Homework Zero}
%%%%%%%%%%%%%%%%%%%%%%%%%%%%%%%%%%%%%%%%%%%%%%%%

\begin{exercise}
(DF \S10.2 Theorem 4): Prove Theorem 4 (Isomorphism Theorems):
	\begin{enumerate}
		\item (\textit{The First Isomorphism Theorem for Modules}) Let $M, N$ be $R$-modules and let $\varphi : M \to N$ be an $R$-modules homomorphism. Then $\ker \varphi$ is a submodule of $M$ and $M/\ker \varphi \cong \varphi(M)$.
			\begin{proof}
				Let $M, N$ be $R$-modules and let $\varphi : M \to N$ be an $R$-modules homomorphism. Then by definition $\varphi(x + y) = \varphi(x) + \varphi(y)$ and $\varphi(rx) = r\varphi(x)$ for all $x, y \in M$, $r \in R$. We want to show that $\ker \varphi = \{m \in M : \varphi(m) = 0\}$ is a submodule. Observe that since $M$ is a module then $M$ is an abelian group by definition so there exists $0 \in M$ such that $m + 0 = m$ for all $m \in M$. In particular $\varphi(0) = \varphi(0 + 0) = \varphi(0) + \varphi(0)$ implying $\varphi(0) = 0$. Conclude that $0 \in \ker \varphi \neq \emptyset$. Now let $r \in R$, $x, y \in \ker \varphi$. Observe that $\varphi(x + ry) = \varphi(x) + \varphi(ry) = \varphi(x) + r\varphi(y) = 0 + r \cdot 0 = 0 + 0 = 0$. Hence $x + ry \in \ker \varphi$. Conclude by the submodule criterion that $\ker \varphi$ is in fact a submodule.\\

				Now define $\Phi : M/\ker \varphi \to \varphi(M)$ by $\Phi(m + \ker \varphi) = \varphi(m)$. We want to show that this mapping is a well-defined bijective homomorphism. We first show well-definedness. Suppose $m + \ker \varphi = m' + \ker \varphi$ it follows by property of cosets that $m - m' \in \ker \varphi$, in particular $\varphi(m - m') = \varphi(m) - \varphi(m') = 0$ and hence $\varphi(m) = \varphi(m')$. But since $\varphi(m) = \Phi(m + \ker \varphi)$ and $\varphi(m') = \Phi(m' + \ker\varphi)$ we have $\Phi(m + \ker\varphi) = \Phi(m' + \ker\varphi)$. Conclude that $\Phi$ is in fact well-defined.\\

				Suppose that $\Phi(m + \ker\varphi) = \Phi(m' + \ker\varphi)$. Then it follows that $\varphi(m) = \varphi(m')$ and so $\varphi(m - m') = 0$ and so $m - m' \in \ker\varphi$. By property of cosets it follows that $m + \ker\varphi = m' + \ker\varphi$ and hence $\Phi$ is injective.\\

				Let $n \in \varphi(M)$. Then by definition of image of $\varphi$ there exists $m \in M$ such that $n = \varphi(m)$. It is immediate that $m + \ker\varphi \in M/\ker\varphi$ and we can conclude that $\Phi$ is surjective.\\

				Now we must show that $\Phi$ is an $R$-module homomorphism. Let $x, y \in M/\ker\varphi$ where $x = m + \ker\varphi$ and $y = m' + \ker\varphi$ for some $m, m' \in M$ and let $r \in R$. Observe that
				\begin{align*}
					\Phi(x + y)&= \Phi(m + m' + \ker\varphi)\\
					&= \varphi(m + m')\\
					&= \varphi(m) + \varphi(m')\\
					&= \Phi(m + \ker\varphi) + \Phi(m' + \ker\varphi)\\
					&= \Phi(x) + \Phi(y)
				\end{align*}
				and
				\begin{align*}
					\Phi(rx) &= \Phi(r(m + \ker\varphi))\\
					&= \Phi(rm + \ker\varphi)\\ %absorbption? is it an ideal?
					&= \varphi(rm)\\
					&= r\varphi(m)\\
					&= r\Phi(m + \ker\varphi)\\
					&= r\Phi(x)
				\end{align*}
				Hence we have shown that $\Phi$ is a well-defined bijective homomorphism and thus we can conclude by definition of $R$-module isomorphism that $M/\ker\varphi \cong \varphi(M)$.
			\end{proof}
		\item (\textit{The Second Isomorphism Theorem}) Let $A, B$ be submodules of the $R$-module $M$. Then $(A + B)/B \cong A/(A \cap B)$.
			\begin{proof}
				Define $\varphi : A \to (A + B)/B$ by $\varphi(a) = a + B$. This mapping is clearly well-defined. We want to show that $\varphi$ is a homomorphism. Let $r \in R$, $a, a' \in A$ and observe that
					\begin{align*}
						\varphi(a + a') &= a + a' + B\\
						&= a + B + a' + B\\
						&= \varphi(a) + \varphi(a')
					\end{align*}
					and
					\begin{align*}
						\varphi(ra) = ra + B\\
						&= r(a + B)\\
						&= r\varphi(a)
					\end{align*}
					and so $\varphi$ is an $R$-module homomorphism by definition. Observe that $\ker \varphi = \{a \in A : \varphi(a) = 0\} = \{a \in A : a + B = 0\} = \{a \in A : a \in B\} = A \cap B$. Now let $x \in (A + B)/B$ then $x = a + b + B$ for some $a \in A$, $b \in B$. But observe that $a + b + B = a + B$ by absorbption. So $\varphi$ is immediately surjective. In particular we have $\varphi(A) = (A + B)/B$. Conclude by the First Isomorphism Theorem for Modules that $A/\ker \varphi = A/(A \cap B) \cong (A + B)/B = \varphi(A)$.
			\end{proof}
		\item (\textit{The Third Isomorphism Theorem}) Let $M$ be an $R$-module, and let $A$ and $B$ be submodules of $M$ with $A \subseteq B$. Then $(M/A)/(B/A) \cong M/B$.
			\begin{proof}
				Define $\varphi M/A \to M/B$ by $\varphi(m + A) = m + B$. We need to show $\varphi$ is well-defined. Suppose $m + A = m' + A$ then $m - m' \in A \subseteq B$ by property of cosets. It also follows that $m + B = m' + B$. Hence $\varphi(m + A) = m + B = m' + B = \varphi(m' + A)$ and hence $\varphi$ is well-defined.\\

				Now we must show $\varphi$ is an $R$-module homomorphism. Let $m, m' \in M$ and $r \in R$. Observe that
				\begin{align*}
					\varphi((m + A) + (m' + a)) &= \varphi(m + m' + A)\\
					&= m + m' + B\\
					&= (m + B) + (m' + B)\\
					&= \varphi(m + A) + \varphi(m' + A)
				\end{align*}
				and
				\begin{align*}
					\varphi(r(m + A)) &= \varphi(rm + A)\\
					&= rm + B\\
					&= r(m + A)\\
					&= r\varphi(m + A)
				\end{align*}
				and hence we can conclude by definition that $\varphi$ is an $R$-module homomorphism.\\

				Observe that $\ker \varphi = \{x \in M/A : \varphi(x) = 0\} = \{m + A " \varphi(m + A) = m + B = 0\} = \{m + A : m \in B\} = B/A$. Let $m + B \in M/B$. Clearly $\varphi(m + A) = m + B$ and hence $\varphi$ is surjective. Now by the First Isomorphism Theorem for Modules we have $(M/A)/\ker \varphi = (M/A)/(B/A) \cong M/B = \varphi(M/A)$.
			\end{proof}
		\item (\textit{The Fourth or Lattice Isomorphism Theorem}) Let $N$ be a submodule of the $R$-module $M$. There is a bijection between the submodules of $M$ which contain $N$ and the submodules of $M/N$. The correspondence is given by $A \leftrightarrow A/N$, for all $A \supseteq N$. The correspondence cummutes with the processes of taking sums and intersections (i.e., is a lattice isomorphism between the lattice of submodules of $M/N$ and the lattice of submodules of $M$ which contain $N$).
			\begin{proof}
				Let $N$ be a submodule of $M$. Define $S = \{K : K$ is a submodule of $M, N \subseteq K\}$, $T = \{L : L$ is a submodule of $M/N\}$. Define $\varphi : S \to T$ by $\varphi(K) = K/N$. We want to show that this mapping is bijective.\\

				Let $K_1, K_2 \in S$ and suppose that $\varphi(K_1) = \varphi(K_2)$. Then $K_1/N = K_2/N$. We want to show that $K_1 = K_2$. Let $x \in K_1$, then $x + N \in K_1/N = K_2/N$, in particular there exists $y \in K_2$ such that $x + N = y + N$. By property of cosets it follows that $x - y \in N$. But since $N \subseteq K_2$ by construction $x - y \in K_2$. Since $K_2$ is a submodule of $M$, it is closed under addition and so $(x - y) + y = x \in K_2$. Conclude that $K_1 \subseteq K_2$. By symmetric argument $K_2 \subseteq K_1$ and hence $K_1 = K_2$. Thus by definition $\varphi$ is injective.\\

				Let $L$ be a submodule of $M/N$. Consider the natural projection map $\pi : M \to M/N$ defined by $\pi(m) = m + N$. We want to show that there exists $K \in S$ such that $\varphi(K) = L$. To do this we will show that $\pi^{-1}(L)$ is a submodule of $M$ and that $N \subseteq \pi^{-1}(L)$. Recall that $\pi^{-1}(L) = \{m \in M : \pi(m) \in L\}$. Observe that $0 \in \pi^{-1}(L)$ since $\pi(0) = 0$ and hence $\pi^{-1}(L) \neq \emptyset$. Let $x, y \in \pi^{-1}(L)$ and $r \in R$. Observe that $\pi(x + ry) = \pi(x) + r \pi(y)$. Since $\pi(y) \in L$ by definition and $L$ is a submodule of $M/N$, it follows that since scalar multiplication is closed $r\pi(y) \in L$. Thus it follows that $\pi(x) + r \pi(y) \in L$ and hence $x + ry \in \pi^{-1}(L)$. Thus by the submodule criterion we can conclude that $\pi^{-1}(L)$ is in fact a submodule. Now let $n \in N$ and observe that $\pi(n) = n + N = 0 + N \in L$ so by definition it follows that $n \in \pi^{-1}(L)$. Conclude that $N \subseteq \pi^{-1}(L)$ and hence $\varphi$ is surjective.\\

				Conclude $\varphi$ is bijective and result follows.
			\end{proof}
	\end{enumerate}
\end{exercise}

\begin{exercise}
(DF \S 10.2 Exercise 1): Use the submodule criterion to show that the kernels and images of $R$-module homomorphisms are submodules.
	\begin{proof}
		Let $M, N$ be $R$-modules and $\varphi : M \to N$ an $R$-module homomorphism. Recall that $\ker \varphi = \{m \in M : \varphi(m) = 0\}$ and $\im \varphi = \{n \in N : $ there exists $m \in M$ with $\varphi(m) = n\}$.\\

		Observe that $\varphi(0) = 0$ so $0 \in \ker \varphi \neq \emptyset$. Let $m, m' \in M$, $r \in R$. Now $\varphi(m + rm') = \varphi(m) + r\varphi(m') = 0 + r \cdot 0 = 0 + 0 = 0$. So $m + rm' \in \ker \varphi$. Thus by the submodule criterion $\ker \varphi$ is a submodule.\\

		Observe that $\varphi(0) = 0 \in N$ so $0 \in \im \varphi \neq \emptyset$. Let $n, n' \in N$, $r \in R$. Then there exists $m, m' \in M$ such that $\varphi(m) = n$ and $\varphi(m') = n'$. Now consider $n + rn'$. $\varphi(m + rm') = \varphi(m) + r\varphi(m') = n + rn'$ so $n + rn' \in \im \varphi$. Conclude by submodule criterion that $\im \varphi$ is in fact a submodule.
	\end{proof}
\end{exercise}

\begin{exercise}
(DF \S 10.2 Exercise 2): Show that the relation ``is $R$-module isomorphic to'' is an equivalence relation on any set of $R$-modules.
	\begin{proof}
		Let $X$ be a set of $R$-modules.
		\begin{itemize}
			\item Let $M \in X$. Observe that $M$ is isomorphic to $M$ trivially. So relation is reflexive.
			\item Let $M, N \in X$. Suppose $M$ is isomorphic to $N$ then by definition there exists $\varphi : M \to N$ that is bijective. Immediately we have $\varphi^{-1} : N \to M$ which is also bijective so $N$ is isomorphic to $M$. By definition the relation is symmetric.
			\item Let $L, M, N \in X$ . Suppose $L$ is isomorphic to $M$, then by definition there exists $\varphi : L \to M$ a bijective $R$-module homomorphism. Suppose $M$ is isomorphic to $N$, then there exists $\Phi : M \to N$ a bijective $R$-module homomorphism. Observe that $\varphi \circ \Phi : L \to N$ is again a bijective $R$-module homomorphism by property of composition of mappings. Hence by definition $L$ is isomorphic to $N$.
		\end{itemize}
		Conclude that the relation ``is $R$-module isomorphic to'' is an equivalence relation on any set of $R$-modules.
	\end{proof}
\end{exercise}

\begin{exercise}
(DF \S 10.2 Exercise 3): Give an explicit example of a map from one $R$-module to another which is a group homomorphism but not an $R$-module homomorphism.
	\begin{solution}
		Consider the Quaternions $\mathbb H = R$; they form a commutative group under addition and a noncommutative group under multiplication. Hence $\mathbb H$ is a noncommutative ring with unity. In particular $\mathbb H$ is an $R$-module over itself. Define $\varphi : \mathbb H \to \mathbb H$ by $\varphi(h) = ih$. This is a group homomorphism since $\varphi(h + h') = i(h + h') = ih + ih' = \varphi(h) + \varphi(h')$. But note that $\varphi(j \cdot 1) = \varphi(j) = ij = k \neq -k = ji = j(i \cdot 1) = j \varphi(1)$. Conclude that $\varphi$ is not an $R$-module homomorphism since the definition is not satisfied.
	\end{solution}
\end{exercise}

\begin{exercise}
(DF \S 10.2 Exercise 4): Let $A$ be and $\Z$-module, let $a$ be any element of $A$ and let $n$ be a positive integer. Prove that the map $\varphi_a : \Z/n\Z \to A$ given by $\varphi(\overline k) = ka$ is a well-defined $\mathbb Z$-module homomorphism if and only if $na = 0$. Prove that $\operatorname{Hom}_\Z(\Z/n\Z, A) \cong A_n$, where $A_n = \{a \in A : na = 0\}$ (So $A_n$ is the annihilator in $A$ of the ideal $(n)$ of $\Z$).
	\begin{proof}
		Suppose that the map $\varphi_a : \Z/n\Z \to A$ given by $\varphi(\overline k) = ka$ is a well-defined $\mathbb Z$-module homomorphism. Then by definition if $\overline m = \overline k$ then $\varphi(\overline m) = \varphi(\overline k)$ or equivalently $ma = ka$. Moreover $\varphi(a + b) = \varphi(a) + \varphi(b)$ and $\varphi(ra) = r\varphi(a)$ for all $a, b \in A$ and $r \in \Z$. Observe that $\overline 0 = \overline k$ so by hypothesis $\varphi(\overline 0) = \varphi(\overline k)$ but observe that $\varphi(\overline 0) = 0 \cdot a = 0$ and $\varphi(\overline k) = ka$. Hence by equality $ka = 0$. Conversely suppose that $na = 0$. We want to show that $\varphi : \Z/n\Z \to A$ defined by $\varphi(\overline k) = ka$ is a well-defined $R$-module homomorphism. Say $\overline k = \overline m$ then by property of cosets $k - m \in \Z/n\Z$ and so $n \cong k - m$. By definition $n \mid k - m$ and hence there exists $t \in \Z$ such that $k - m = nt$. Observe that
		\begin{align*}
			k - m &= nt\\
			(k - m)a &= nta\\
			ka - ma &= (na)t\\
			ka - ma &= 0\\
			ka &= ma 
		\end{align*}
		Thus we have $\varphi(\overline k) = \varphi(\overline m)$ and we can conclude that $\varphi$ is in fact a well-defined $R$-module homomorphism.\\

		Now we want to show that $\operatorname{Hom}_\Z(\Z/n\Z, A) \cong A_n = \{a \in A : na = 0\}$. \textit{Note}: We are making the assumption that we want to show this in an isomorphism of $R$-modules as exercise does not specifiy group, ring or module isomorphism. Define $\Phi : A_n \to \operatorname{Hom}_\Z(\Z/n\Z, A)$ by $\Phi(a) = \varphi_a$. We will show that this is in fact an $R$-module homomorphism, then that is a bijection.\\

		Let $a, a' \in A_n$, $r \in \Z$. Observe that
		\begin{align*}
			\Phi(a + a')(\overline k)&= \varphi_{a + a'}(\overline k)\\
			&= (a + a')k\\
			&= ak + a'k\\
			&= \varphi_a(\overline k) + \varphi_{a'}(\overline k)\\
			&= \Phi(a)(\overline k) + \Phi(a')(\overline k)
		\end{align*}
		So $\Phi(a + a') = \Phi(a) + \Phi(a')$ by definition. Moreover
		\begin{align*}
			\Phi(ra)(\overline k) &= \varphi_{ra}(\overline k)\\
			&= rak\\
			&= r\varphi_{a}(\overline k)\\
			&= r\Phi(a)(\overline k)
		\end{align*}
		Hence $\Phi(ra) = r\Phi(a)$ by definition. Conclude by definition that $\Phi$ is in fact an $R$-module homomorphism.\\

		Recall that $\ker \Phi = \{a \in A_n : \Phi(a) = 0\}$ and observe that
		\begin{align*}
			\ker \varphi &= \{a \in A_n : \Phi(a) = 0\}\\
			&= \{a \in A_n : \varphi_a(\overline k) = 0 \text{ for all k } \in \Z/n\Z\}\\
			&= \{0\}
		\end{align*}
		So we conclude that $\ker \Phi = \{0\}$ and hence $\Phi$ is injective.\\

		Let $\varphi \in \operatorname{Hom}_\Z(\Z/n\Z, A_n)$. Define $a = \varphi(\overline 1)$ and hence $na = n\varphi(\overline 1) = \varphi(n \overline 1) = \varphi(\overline n) = \varphi(\overline 0) = 0$ and hence $a \in A_n$. Observe that for all $\overline k \in \Z/n\Z$ it follows that $\varphi(\overline k) = \varphi(\overline k \cdot \overline 1) = \varphi(k \overline 1) = k \varphi(\overline 1) = ka = \varphi_a(\overline k)$. Thus by definition $\varphi = \varphi_a$. So for any $\varphi \in \operatorname{Hom}(\Z/n\Z, A_n)$ we can find $a \in A_n$ such that $\Phi(a) = \varphi_a = \varphi$. Conclude by definition that $\Phi$ is surjective.\\

		Now applying the First Isomorphism Theorem for $R$-modules we can conclude that $\operatorname{Hom}(\Z/n\Z, A_n) \cong A_n$.
	\end{proof}
\end{exercise}

\begin{exercise}
(DF \S 10.2 Exercise 5): Exhibit all $\Z$-module homomorphisms from $\Z/30\Z$ to $\Z/21\Z$.
	By previous exercise we know $\operatorname{Hom}(\Z/30\Z, \Z/21\Z) \cong (\Z/21\Z)_{30}$ where $(\Z/21\Z)_{30} = \{a \in \Z/21\Z : 30a = 0\} = A_{30}$. Observe that $(\Z/21\Z)_{30} = 7\Z/30\Z = \{0, 7, 14\}$ and it has three elements. Hence we know that there are three homomorphisms from $\Z/30\Z$ to $\Z/21\Z$. The three homomorphisms are the ones defined by the trivial homomorphism, $\varphi_7(\overline x) = 7x$, $\varphi_{14}(\overline x) = 14x$.
\end{exercise}

\begin{exercise}
(DF \S 10.2 Exercise 6): Prove that $\operatorname{Hom}_\Z(\Z/n\Z, \Z/m\Z) \cong \Z/(n, m)\Z$.
\end{exercise}

\begin{exercise}
(DF \S 10.2 Exercise 7): Let $z$ be a fixed element of the center of $R$. Prove that the map $m \to zm$ is an $R$-module homomorphism from $M$ to itself. Show that for a commutative ring $R$ the map from $R$ to $\operatorname{End}_R{M}$ given by $r \to rI$ is a ring homomorphism (where $I$ is the identity endomorphism).)
	\begin{proof}
		Recall that the center of $R$ is $\{z \in R : zr = rz$ for all $r \in R\}$. Let $z$ be in the center of $R$ then by definition $zr = rz$ for all $r \in R$. Define $\varphi : M \to M$ by $\varphi(m) = zm$. We claim that this is an $R$-module homomorphism. Let $m, m' \in M$, $r \in R$. Observe that $\varphi(m + m') = z(m + m') zm + zm' = \varphi(m) + \varphi(m')$ and $\varphi(rm) = zrm = rzm = r\varphi(m)$. Conclude by definition that $\varphi$ is in fact an $R$-module homomorphism.\\

		\textit{Note:} We are making the assumption that ``$I$ being the identity endomorphism'' means \textit{multiplicative identity}. Now let $R$ be a commutative ring and define $\Phi : R \to \operatorname{End}_R(M)$ by $\Phi(r) = rI$ where $I : M \to M$, defined by $I(m) = m$, is the identity endomorphism. We want to show that $\Phi$ is a ring homomorphism. Let $r, s \in R$ and observe that for all $m \in M$
		\begin{align*}
			\Phi(r + s)(m) &= (r + s)I(m)\\
			&= (r + s)m\\
			&= rm + rs\\
			&= rI(m) + sI(m)\\
			&= \Phi(r)(m) + \Phi(s)(m)
		\end{align*}
		So by definition $\Phi(r + s) = \Phi(r) + \Phi(s)$. Moreover,
		\begin{align*}
			\Phi(rs) &= rsI(m)\\
			&= rsI(m)I(m)\\
			&= rI(m) \cdot sI(m)\\
			&= \Phi(r)(m) \Phi(s)(m)
		\end{align*}
		And hence $\Phi(rs) = \Phi(r)\Phi(s)$ and we can conclude by definition that $\Phi$ is a ring homomorphism.
	\end{proof}
\end{exercise}

\begin{exercise}
(DF \S 10.2 Exercise 8): Let $\varphi : M \to N$ be an $R$-module homomorphism. Prove that $\varphi(\operatorname{Tor}(M)) \subseteq \operatorname{Tor}(N)$.
	\begin{proof}
		Recall that $\operatorname{Tor}(M) = \{m \in M : rm = 0$ for some $0 \neq r \in R\}$. Now it follows that $\varphi(\operatorname{Tor}(M)) = \{n \in N : n = \varphi(m)$ for some $m \in \operatorname{Tor}(M)\}$. Let $n \in \varphi(\operatorname{Tor}(M))$ then $n = \varphi(m)$ for some $m \in \operatorname{Tor}(M)$ by definition. Since $m \in \operatorname{Tor}(M)$ there exists $0 \neq r \in R$ such that $rm = 0$. Hence $rn = r \varphi(m) = \varphi(rm) = \varphi(0) = 0$. Conclude that $n \in \operatorname{Tor}(N)$ and hence $\varphi(\operatorname{Tor}(M)) \subseteq \operatorname{Tor}(N)$.
	\end{proof}
\end{exercise}

\begin{exercise}
(DF \S 10.2 Exercise 9): Let $R$ be a commutative ring. Prove that $\operatorname{Hom}_R(R, M)$ and $M$ are isomorphic as left $R$-modules.
	\begin{proof}
		Define $\Phi : \operatorname{Hom}_R(R, M) \to M$ by $\Phi(\varphi) = \varphi(1)$. We must first show that this is in fact an $R$-module homomorphism. Observe that for all $\varphi, \xi \in \operatorname{Hom}_R(R, M)$ and all $r \in R$ it follows that
		\begin{align*}
			\Phi(\varphi + \xi) &= = (\varphi + \xi)(1)\\
			&= \varphi(1) + \varphi(\xi)\\
			&= \Phi(\varphi) + \Phi(\xi) 
		\end{align*}
		and also by Proposition 2 we have
		\begin{align*}
			\Phi(r\varphi) &= (r\varphi)(1)\\
			&= r \varphi(1)\\
			&= r \Phi(\varphi)
		\end{align*}
		Hence we can now conclude that $\Phi$ is an $R$-module homomorphism.\\

		We must now show that $\Phi$ is injective. Suppose that $\Phi(\varphi) = \Phi(\xi)$ then by definition $\varphi(1) = \xi(1)$ or equivalently $\varphi(1) - \xi(1) = 0$ and hence $(\varphi - \xi)(1) = 0$. But since $\varphi - \xi \in \operatorname{Hom}_R(R, M)$ it is an $R$-module homomorphism so $(\varphi - \xi)(x) = (\varphi - \xi)(x \cdot 1) = x(\varphi - \xi)(1) = x \cdot 0 = 0$ for all $x \in R$. Conclude that $\varphi(x) = \xi(x)$ for all $x \in R$ and hence by definition $\varphi = \xi$. Conclude that $\Phi$ is inective.\\

		We must now show that $\Phi$ is surjective. Let $m \in M$ be arbitrary. We want to show that there exists $\varphi \in \operatorname{Hom}_R(R, M)$ such that $\Phi(\varphi) = m$. Let us define $\varphi : R \to M$ by $\varphi(x) = xm$. We need to show that $\varphi \in \operatorname{Hom}_R(R, M)$. Observe that $\varphi(x + y) = (x + y)m = xm + ym = \varphi(x) + \varphi(y)$ for all $x, y \in R$ and $\varphi(rx) = rxm = r \varphi(x)$ for all $x \in R$, $r \in R$. Hence we have shown that $\varphi$ is in fact an $R$-module homomorphism. Now observe that $\Phi(\varphi) = \varphi(1) = 1 \cdot m = m$. Conclude by definition that $\Phi$ is surjective.\\

	We have shown that $\Phi$ is a bijective $R$-module homomorphism. Conclude that $\operatorname{Hom}_R(R, M) \cong M$.	
	\end{proof}
\end{exercise}

\begin{exercise}
(DF \S 10.2 Exercise 10): Let $R$ be a commutative ring. Prove that $\operatorname{Hom}_R(R, R)$ and $R$ are isomorphic as rings.
	\begin{proof}
		Define $\Phi : \operatorname{Hom}_R(R, R) \to R$ by $\Phi(\varphi) = \varphi(1)$. We will first show that $\Phi$ is a ring homomorphism. Observe that for all $\varphi, \xi \in \operatorname{Hom}_R(R, R)$ and all $r \in R$ by Proposition 2 we have
		\begin{align*}
			\Phi(\varphi + \xi) &= (\varphi + \xi)(1)\\
			&= \varphi(1) + \xi(1)\\
			&= \Phi(\varphi) + \Phi(\xi)
		\end{align*}
		and also by property of commutativity
		\begin{align*}
			\Phi(\varphi \circ \xi) &= (\varphi \circ \xi)(1)\\
			&= \varphi(\xi(1))\\
			&= \varphi(\xi(1) \cdot 1)\\
			&= \xi(1) \varphi(1)\\
			&= \varphi(1) \xi(1)\\
			&= \Phi(\varphi) \Phi(\xi)
		\end{align*}
		Hence we can conclude that $\Phi$ is in fact a ring homomorphism.\\

		We must now show that $\Phi$ is injective. Suppose that $\Phi(\varphi) = \Phi(\xi)$ then by definition $\varphi(1) = \xi(1)$ or equivalently $\varphi(1) - \xi(1) = 0$ and hence $(\varphi - \xi)(1) = 0$. But since $\varphi - \xi \in \operatorname{Hom}_R(R, R)$ it is an $R$-module homomorphism so $(\varphi - \xi)(x) = (\varphi - \xi)(x \cdot 1) = x(\varphi - \xi)(1) = x \cdot 0 = 0$ for all $x \in R$. Conclude that $\varphi(x) = \xi(x)$ for all $x \in R$ and hence by definition $\varphi = \xi$. Conclude that $\Phi$ is inective.\\

		We must now show that $\Phi$ is surjective. Let $r \in R$ be arbitrary. We want to show that there exists $\varphi \in \operatorname{Hom}_R(R, R)$ such that $\Phi(\varphi) = r$. Let us define $\varphi : R \to R$ by $\varphi(x) = xr$. We need to show that $\varphi \in \operatorname{Hom}_R(R, R)$. Observe that $\varphi(x + y) = (x + y)r = xr + yr = \varphi(x) + \varphi(y)$ for all $x, y \in R$ and $\varphi(sx) = sxr = s\varphi(x)$ for all $s \in R$ and all $x \in R$. Hence we have shown that $\varphi$ is in fact an $R$-module homomorphism. Now observe that $\Phi(\varphi) = \varphi(1) = 1 \cdot r = r$. Conclude by definition that $\Phi$ is surjective.\\

		We have shown that $\Phi$ is a bijective ring homomorphism. Conclude that $\operatorname{Hom}_R(R, R) \cong R$.
	\end{proof}
\end{exercise}

\begin{exercise}
(DF \S 10.2 Exercise 11): Let $A_1, A_2, \ldots, A_n$ be $R$-modules and let $B_i$ be submodules of $A_i$ for each $i = 1, 2, \ldots, n$. Prove that $$(A_1 \times A_2 \times \cdots \times A_n)/(B_1 \times B_2 \times \cdots \times B_n) \cong (A_1/B_1) \times (A_2/B_2) \times \cdots \times (A_n/B_n).$$
	\begin{proof}
		Define $\varphi : A_1 \times A_2 \times \cdots \times A_n \to (A_1/B_1) \times (A_2/B_2) \times \cdots \times (A_n/B_n)$ by $\varphi(a_1, a_2, \ldots, a_n) = (a_1 + B_1, a_2 + B_2, \ldots, a_n + B_n)$. We want to show that this is an $R$-module homomorphism and that $\ker \varphi = B_1 \times B_2 \times \cdots \times B_n$ and that the mapping is surjective. Then the first isomorphism theorem for modules yields the result.\\

		Let $x, y \in A_1 \times A_2 \times \cdots \times A_n$ where $x = (a_1, a_2, \ldots, a_n)$ and $y = (a_1', a_2', \ldots, a_n')$. Observe that
		\begin{align*}
			\varphi(x + y) &= \varphi((a_1, a_2, \ldots, a_n) + (a_1', a_2', \ldots, a_n'))\\
			&= \varphi(a_1 + a_1', a_2 + a_2', \ldots, a_n + a_n')\\
			&= (a_1 + a_1' + B_1, a_2 + a_2' + B_2, \ldots, a_n + a_n' + B_n)\\
			&= (a_1 + B_1 + a_1' + B_1, a_2 + B_2 + a_2' + B_2, \ldots, a_n + B_n + a_n' + B_n)\\
			&= (a_1 + B_1, a_2 + B_2, \ldots, a_n + B_n) + (a_1' + B_1, a_2' + B_2, \ldots, a_n' + B_n)\\
			&= \varphi(a_1, a_2, \ldots, a_n) + \varphi(a_1', a_2', \ldots, a_n')\\
			&= \varphi(x) + \varphi(y)
		\end{align*}
		and for $r \in R$
		\begin{align*}
			\varphi(rx) &= \varphi(r(a_1, a_2, \ldots, a_n))\\
			&= \varphi(ra_1, ra_2, \ldots, ra_n)\\
			&= (ra_1 + B_1, ra_2 + B_2, \ldots, ra_n + B_n)\\
			&= (r(a_1 + B_1), r(a_2 + B_2), \ldots, r(a_n + B_n))\\
			&= r(a_1 + B_1, a_2 + B_2, \ldots, a_n + B_n)\\
			&= r\varphi(a_1, a_2, \ldots, a_n)\\
			&= r\varphi(x)
		\end{align*}
		Thus by definition we can conclude that $\varphi$ is an $R$-moduel homomorphism.\\

		Now we want to show that $\ker \varphi = B_1 \times B_2 \times \cdots \times B_n$. Observe that
		\begin{align*}
			\ker \varphi &= \{x \in A_1 \times A_2 \times \cdots \times A_n : \varphi(x)= 0\}\\
			&= \{(a_1, a_2, \ldots, a_n) \in A_1 \times A_2 \times \cdots \times A_n : \varphi(a_1, a_2, \ldots, a_n) = 0\}\\
			&= \{(a_1, a_2, \ldots, a_n) \in A_1 \times A_2 \times \cdots \times A_n : (a_1 + B_1, a_2 + B_2, \ldots, a_n + B_n) = (0, 0, \ldots, 0)\}\\
			&= \{(a_1, a_2, \ldots, a_n) \in A_1 \times A_2 \times \cdots \times A_n : a_1 \in B_1, a_2 \in B_2, \ldots, a_n \in B_n\}
		\end{align*}
		Hence $\ker\varphi \subseteq B_1 \times B_2 \times \cdots \times B_n$ and trivially $B_1 \times B_2 \times \cdots \times B_n \subseteq \ker \varphi$ by construction of $\varphi$. Conclude that $\ker \varphi = B_1 \times B_2 \times \cdots \times B_n$.\\

		The mapping is trivially surjective. Applying first Isomorphism theorem yields the result.
	\end{proof}
\end{exercise}

\begin{exercise}
(DF \S 10.3 Exercise 3): Show that the $F[x]$-modules in Exercises 18 and 19 of Section 1 are both cyclic.
\end{exercise}

\begin{exercise}
(DF \S 10.3 Exercise 4): An $R$-module $M$ is called a \textit{torsion} module if for each $m \in M$ there is a nonzero element $r \in R$ such that $rm = 0$, where $r$ may depend on $m$ (i.e., $M = \operatorname{Tor}(M)$ in the notation of Exercise 8 of Section 1). Prove that every finite abelian group is a torsion $\Z$-module. Give an example of an infinite abelian group that is a torsion $\Z$-module.
\end{exercise}

\begin{exercise}
(DF \S 10.3 Exercise 5): Let $R$ be an intergral domain. Prove that every finitely generated torsion $R$-module has a nonzero annihilator i.e., there is a nonzero element $r \in R$ such that $rm = 0$ for all $m \in M$ -- here $r$ does not depend on $m$ (the annihilator of a module was defined in Exercise 9 of Section 1). Give an example of a torsion $R$-module whose annihilator is the zero ideal.
\end{exercise}

\begin{exercise}
(DF \S 10.3 Exercise 9): An $R$-module $M$ is called \textit{irreducible} if $M \neq 0$ and if $0$ and $M$ are the only submodules of $M$. Show that $M$ is irreducible if and only if $M \neq 0$ and $M$ is a cyclic module with any nonzero element as its generator. Determine all the irreducible $\Z$-modules.
\end{exercise}

\begin{exercise}
(DF \S 10.3 Exercise 10): Assume $R$ is commutative. Show that an $R$-module $M$ is irreducible if and only if $M$ is isomorphic (as an $R$-module) to $R/I$ where $I$ is a maximal ideal of $R$. [By the previous exercise, if $M$ is irreducible there is a natural map $R \to M$ defined by $r \mapsto rm$, where $m$ is any fixed nonzero element of $M$.]
\end{exercise}

\begin{exercise}
(DF \S 10.3 Exercise 15): An element $e \in R$ is called a \textit{central idempotent} if $e^2 = e$ and $er = re$ for all $r \in R$. If $e$ is a central idempotent in $R$, prove that $M = eM \oplus (1 - e)M$. [Recall Exercise 14 in Section 1.]
\end{exercise}

\begin{exercise}
(DF \S 10.3 Exercise 16): For any ideal $I$ of $R$ let $IM$ be the submodule defined in Exercise 5 of Section 1. Let $A_1, \ldots, A_k$ be any ideals in the ring $R$. Prove that the map $$M \to M/A_1M \times \cdots \times M/A_kM \text{ defined by } m \mapsto (m + A_1M, \ldots, m + A_kM)$$ is an $R$-module homomorphism with kernel $A_1M \cap A_2M \cap \cdots \cap A_kM$.
\end{exercise}

\begin{exercise}
(DF \S 10.3 Exercise 22): Let $R$ be a Principal Ideal Domain, let $M$ be a torsion $R$-module (cf. Exercise 4) and let $p$ be a prime in $R$ (do not assume $M$ is finitely generated, hence it need not have a nonzero annihilator -- cd. Exercise 5). The \textit{$p$-primary component} of $M$ is the set of all elements of $M$ that are annihilated by some positive power of $p$.
	\begin{enumerate}
		\item Prove that the $p$-primary component is a submodule. [See Exercise 13 in Section 1.]
		\item Prove that this definition of $p$-primary component agrees with the one given in Exercise 18 when $M$ has a nonzero annihilator.
		\item Prove that $M$ is the (possibly infinite) direct sum of its $p$-primary components, as $p$ runs over all primes of $R$.
	\end{enumerate}
\end{exercise}

\end{document}




