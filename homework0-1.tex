\documentclass[8pt]{amsart}
\usepackage{amsmath, amssymb}
\usepackage{amsfonts}
\usepackage{mathrsfs}
\usepackage[arrow,matrix,curve,cmtip,ps]{xy}
\usepackage{paralist}
\usepackage[left=1in,top=1in,right=1in,bottom=1in]{geometry}
\usepackage{amsthm}
\usepackage{color}

\allowdisplaybreaks

\theoremstyle{plain}% default

\theoremstyle{definition}
\newtheorem{theorem}{Theorem}[section]
\newtheorem{lemma}{Lemma}[section]
\newtheorem*{proposition}{Proposition}
\newtheorem*{corollary}{Corollary}
\newtheorem*{KL}{Klein’s Lemma}

\newtheorem*{definition}{Definition}
\newtheorem{conjecture}{Conjecture}[section]
\newtheorem{example}{Example}[section]
\newtheorem*{exercise}{Exercise}%[section]
\newtheorem*{notation}{Notation}
\newtheorem*{remark}{Remark}

\theoremstyle{remark}
\newtheorem*{note}{Note}
\newtheorem{case}{Case}

%this has equations numbered within sections 1.1,1.2, ... 2.1,...
\numberwithin{equation}{section}

\makeatletter
\newenvironment{solution}
               {\let\oldqedsymbol=\qedsymbol%
                \def\@addpunct##1{}%
                \renewcommand{\qedsymbol}{$\blacktriangleleft$}%
                \begin{proof}[\itshape Solution.]}%
               {\end{proof}%
                \renewcommand{\qedsymbol}{\oldqedsymbol}}
\makeatother

\def\upint{\mathchoice%
    {\mkern13mu\overline{\vphantom{\intop}\mkern7mu}\mkern-20mu}%
    {\mkern7mu\overline{\vphantom{\intop}\mkern7mu}\mkern-14mu}%
    {\mkern7mu\overline{\vphantom{\intop}\mkern7mu}\mkern-14mu}%
    {\mkern7mu\overline{\vphantom{\intop}\mkern7mu}\mkern-14mu}%
  \int}
\def\lowint{\mkern3mu\underline{\vphantom{\intop}\mkern7mu}\mkern-10mu\int}


%-------------------------------------------
%       Begin Local Macros
%-------------------------------------------

\newcommand{\Z}{\mathbb{Z}}
\newcommand{\N}{\mathbb{N}}
\newcommand{\Q}{\mathbb{Q}}
\newcommand{\R}{\mathbb{R}}
\newcommand{\C}{\mathbb{C}}
\newcommand{\T}{\mathbb{T}}
\newcommand{\D}{\displaystyle}
\newcommand{\im}{\operatorname{im}}
\newcommand{\coker}{\operatorname{coker}}
\newcommand{\ind}{\operatorname{ind}}
\newcommand{\rank}{\operatorname{rank}}
\newcommand\mc[1]{\marginpar{\sloppy\protect\footnotesize #1}}

%-------------------------------------------
%       End Local Macros
%-------------------------------------------
\begin{document}
\title[MATH 697]{Introduction to Category Theory}

% \author{Robert Cardona}

\author{
	Robert Cardona %\textit{mrrobertcardona@gmail.com}
	\and
	Massy Khoshbin %\textit{massy255@gmail.com}
	\and
	Siavash Mortezavi %\textit{siavash.mortezavi@gmail.com}
}


\address{Department of Mathematics \\ California State University Long Beach}
\email{mrrobertcardona@gmail.com \and massy255@gmail.com \and siavash.mortezavi@gmail.com}

\date{\today}


\maketitle

%%%%%%%%%%%%%%%%%%%%%%%%%%%%%%%%%%%%%%%%%%%%%%%%
\setcounter{section}{-1}
\section{MATH 697 Homework Zero.One}
%%%%%%%%%%%%%%%%%%%%%%%%%%%%%%%%%%%%%%%%%%%%%%%%


\textbf{\S10.2 Theorem 4: (Isomorphism Theorems)}:
	\begin{enumerate}
		\item (\textit{The First Isomorphism Theorem for Modules}) Let $M, N$ be $R$-modules and let $\varphi : M \to N$ be an $R$-modules homomorphism. Then $\ker \varphi$ is a submodule of $M$ and $M/\ker \varphi \cong \varphi(M)$.
			\begin{proof}
				$\varphi$ is, in particular, a group homomorphism from $M$ to $N$. By First Isomorphism Theorem for groups, Ker$\varphi \trianglelefteq M$ and $\exists$ group isomorphism $\phi : M/\ker \varphi \to \varphi(M)$ satisfying $\phi(\overline{m})$ = $\varphi(m)$. Since $\varphi$ is an $R$-module homomorphism, for $r \in R$, have $\phi(r\overline{m}) = \phi(\overline{rm}) = \varphi(rm) = r\varphi(m) = r\phi(\overline{m})$. Thus $\phi$ is an $R$-module isomorphism.  
			\end{proof}
		\item (\textit{The Second Isomorphism Theorem}) Let $A, B$ be submodules of the $R$-module $M$. Then $(A + B)/B \cong A/(A \cap B)$.
			\begin{proof}
				Define $\varphi : A \to (A + B)/B$ by $\varphi(a) = a + B$. By the Second Isomorphism Theorem for groups, $\varphi$ is a group homomorphism. Let $r \in R$, then
				
				\begin{align*}
						\varphi(ra) &= ra + B\\
						&{\color{red} = ra + rB }\\
						&= r(a + B)\\
						&= r\varphi(a)
					\end{align*}
					and so $\varphi$ is an $R$-module homomorphism by definition. Observe that $\ker \varphi = \{a \in A : \varphi(a) = 0\} = \{a \in A : a + B = 0\} = \{a \in A : a \in B\} = A \cap B$. Now let $x \in (A + B)/B$ then $x = a + b + B$ for some $a \in A$, $b \in B$. But observe that $a + b + B = a + B$ by absorbption. So $\varphi$ is immediately surjective. In particular we have $\varphi(A) = (A + B)/B$. By the First Isomorphism Theorem for Modules, $A/\ker \varphi = A/(A \cap B) \cong (A + B)/B = \varphi(A)$.
			\end{proof}
		\item (\textit{The Third Isomorphism Theorem}) Let $M$ be an $R$-module, and let $A$ and $B$ be submodules of $M$ with $A \subseteq B$. Then $(M/A)/(B/A) \cong M/B$.
			\begin{proof}
				Define $\varphi M/A \to M/B$ by $\varphi(m + A) = m + B$. By the Third Isomorphism Theorem for groups, $\varphi$ is a group homomorphism. Let $r \in R$, then
				
				\begin{align*}
					\varphi(r(m + A)) &= \varphi(rm + rA)\\
					&= r(m + A)\\
					&= r\varphi(m + A)
				\end{align*}
				
				and thus $\varphi$ is an $R$-module homomorphism.\\

				Observe that $\ker \varphi = \{x \in M/A : \varphi(x) = 0\} = \{m + A " \varphi(m + A) = m + B = 0\} = \{m + A : m \in B\} = B/A$. Let $m + B \in M/B$. Clearly $\varphi(m + A) = m + B$ and hence $\varphi$ is surjective. Now by the First Isomorphism Theorem for Modules we have $(M/A)/\ker \varphi = (M/A)/(B/A) \cong M/B = \varphi(M/A)$.
			\end{proof}
		\item (\textit{The Fourth or Lattice Isomorphism Theorem}) Let $N$ be a submodule of the $R$-module $M$. There is a bijection between the submodules of $M$ which contain $N$ and the submodules of $M/N$. The correspondence is given by $A \leftrightarrow A/N$, for all $A \supseteq N$. The correspondence cummutes with the processes of taking sums and intersections (i.e., is a lattice isomorphism between the lattice of submodules of $M/N$ and the lattice of submodules of $M$ which contain $N$).
			\begin{proof}
				Let $N$ be a submodule of $M$. Define $S = \{K : K$ is a submodule of $M, N \subseteq K\}$, $T = \{L : L$ is a submodule of $M/N\}$. Define $\varphi : S \to T$ by $\varphi(K) = K/N$. We want to show that this mapping is bijective.\\

				Let $K_1, K_2 \in S$ and suppose that $\varphi(K_1) = \varphi(K_2)$. Then $K_1/N = K_2/N$. We want to show that $K_1 = K_2$. Let $x \in K_1$, then $x + N \in K_1/N = K_2/N$, in particular there exists $y \in K_2$ such that $x + N = y + N$. By property of cosets it follows that $x - y \in N$. But since $N \subseteq K_2$ by construction $x - y \in K_2$. Since $K_2$ is a submodule of $M$, it is closed under addition and so $(x - y) + y = x \in K_2$. Conclude that $K_1 \subseteq K_2$. By symmetric argument $K_2 \subseteq K_1$ and hence $K_1 = K_2$. Thus by definition $\varphi$ is injective.\\

				Let $L$ be a submodule of $M/N$. Consider the natural projection map $\pi : M \to M/N$ defined by $\pi(m) = m + N$. We want to show that there exists $K \in S$ such that $\varphi(K) = L$. To do this we will show that $\pi^{-1}(L)$ is a submodule of $M$ and that $N \subseteq \pi^{-1}(L)$. Recall that $\pi^{-1}(L) = \{m \in M : \pi(m) \in L\}$. Observe that $0 \in \pi^{-1}(L)$ since $\pi(0) = 0$ and hence $\pi^{-1}(L) \neq \emptyset$. Let $x, y \in \pi^{-1}(L)$ and $r \in R$. Observe that $\pi(x + ry) = \pi(x) + r \pi(y)$. Since $\pi(y) \in L$ by definition and $L$ is a submodule of $M/N$, it follows that since scalar multiplication is closed $r\pi(y) \in L$. Thus it follows that $\pi(x) + r \pi(y) \in L$ and hence $x + ry \in \pi^{-1}(L)$. Thus $\pi^{-1}(L)$ is a submodule. Now let $n \in N$ and observe that $\pi(n) = n + N = 0 + N \in L$ so by definition it follows that $n \in \pi^{-1}(L)$. Conclude that $N \subseteq \pi^{-1}(L)$ and hence $\varphi$ is surjective.\\

				Conclude $\varphi$ is bijective and result follows.
			\end{proof}
	\end{enumerate}



\textbf{\S 10.2 \#1}: Use the submodule criterion to show that the kernels and images of $R$-module homomorphisms are submodules.
	\begin{proof}
		Let $M, N$ be $R$-modules and $\varphi : M \to N$ an $R$-module homomorphism. Recall that $\ker \varphi = \{m \in M : \varphi(m) = 0\}$ and $\im \varphi = \{n \in N : $ there exists $m \in M$ with $\varphi(m) = n\}$.\\

		Observe that $\varphi(0) = 0$ so $0 \in \ker \varphi \neq \emptyset$. Let $m, m' \in M$, $r \in R$. Now $\varphi(m + rm') = \varphi(m) + r\varphi(m') = 0 + r \cdot 0 = 0 + 0 = 0$. So $m + rm' \in \ker \varphi$. Thus by the submodule criterion $\ker \varphi$ is a submodule.\\

		Observe that $\varphi(0) = 0 \in N$ so $0 \in \im \varphi \neq \emptyset$. Let $n, n' \in N$, $r \in R$. Then there exists $m, m' \in M$ such that $\varphi(m) = n$ and $\varphi(m') = n'$. Now consider $n + rn'$. $\varphi(m + rm') = \varphi(m) + r\varphi(m') = n + rn'$ so $n + rn' \in \im \varphi$. Conclude that $\im \varphi$ is a submodule.
	\end{proof}


\textbf{\S 10.2 \#2}: Show that the relation ``is $R$-module isomorphic to'' is an equivalence relation on any set of $R$-modules.
	\begin{proof}
		Let $X$ be a set of $R$-modules.
		\begin{itemize}
			\item {\color{red} Let $M$ be an $R$-module. Define $\varphi : M \to M$ by $\varphi(m) = m$. Observe that $\varphi(rm  + sn) = rm + sn = r\varphi(m) + s\varphi(n)$ so it is an $R$-module homomorphism. If $\varphi(m) = 0$ then $m = 0$ and thus by definition $\varphi$ is injective. Choose $m \in M$, immediately $\varphi(m) = m$ so $\varphi$ is injective. Since $\varphi$ is a bijective $R$-module homomorphism, conclude  that $M$ is isomorphic to $M$. So relation is reflexive.}
			\item Let $M, N \in X$. Suppose $M$ is isomorphic to $N$ then by definition there exists {\color{red} an $R$-module homomorphism} $\varphi : M \to N$ that is bijective. Immediately we have {\color{red} its inverse by bijectivity} $\varphi^{-1} : N \to M$ which is also bijective so $N$ is isomorphic to $M$. By definition the relation is symmetric.
			\item Let $L, M, N \in X$ . Suppose $L$ is isomorphic to $M$, then by definition there exists $\varphi : L \to M$ a bijective $R$-module homomorphism. Suppose $M$ is isomorphic to $N$, then there exists $\Phi : M \to N$ a bijective $R$-module homomorphism. Observe that $\varphi \circ \Phi : L \to N$ is again a bijective $R$-module homomorphism by property of composition of mappings. Hence by definition $L$ is isomorphic to $N$.
		\end{itemize}
		Conclude that the relation ``is $R$-module isomorphic to'' is an equivalence relation on any set of $R$-modules.
	\end{proof}



\textbf{\S 10.2 \#3}: Give an explicit example of a map from one $R$-module to another which is a group homomorphism but not an $R$-module homomorphism.
	\begin{solution}
		Consider the Quaternions $\mathbb H = R$; they form a commutative group under addition \textit{and a noncommutative group under multiplication}. ({ \color{green} Question: You commented in the assignment: ``Not Quite", can you explain?}) Hence $\mathbb H$ is a noncommutative ring with unity. In particular $\mathbb H$ is an $R$-module over itself. Define $\varphi : \mathbb H \to \mathbb H$ by $\varphi(h) = ih$. This is a group homomorphism since $\varphi(h + h') = i(h + h') = ih + ih' = \varphi(h) + \varphi(h')$. But note that $\varphi(j \cdot 1) = \varphi(j) = ij = k \neq -k = ji = j(i \cdot 1) = j \varphi(1)$. Conclude that $\varphi$ is not an $R$-module homomorphism since the definition is not satisfied.\\

		{\color{red} For a commutative example, consider $\R[x]$ as a module over iteself. Define $\varphi : \R[x] \to \R[x]$ by $\varphi(f(x)) = f(x^2)$. Observe that
		\begin{align*}
			\varphi(f(x) + g(x)) &= \varphi((f + g)(x))\\
			&= (f + g)(x^2)\\
			&= f(x^2) + g(x^2)\\
			&= \varphi(f(x)) + \varphi(g(x))
		\end{align*}
		and so $\varphi$ is a group homomorphism, but observe that $$x\varphi(f(x)) = xf(x^2) \neq x^2f(x^2) = \varphi(xf(x)).$$ which implies that $\varphi$ is not an $R$-module homomorphism.}
	\end{solution}


\textbf{\S 10.2 \#4}: Let $A$ be a $\Z$-module, let $a$ be any element of $A$ and let $n$ be a positive integer. Prove that the map $\varphi_a : \Z/n\Z \to A$ given by $\varphi(\overline k) = ka$ is a well-defined $\mathbb Z$-module homomorphism if and only if $na = 0$. Prove that $\operatorname{Hom}_\Z(\Z/n\Z, A) \cong A_n$, where $A_n = \{a \in A : na = 0\}$ (So $A_n$ is the annihilator in $A$ of the ideal $(n)$ of $\Z$).
	\begin{proof}
		Suppose that the map $\varphi_a : \Z/n\Z \to A$ given by $\varphi(\overline k) = ka$ is a well-defined $\mathbb Z$-module homomorphism. Then by definition if $\overline m = \overline k$ then $\varphi(\overline m) = \varphi(\overline k)$ or equivalently $ma = ka$. Moreover $\varphi(a + b) = \varphi(a) + \varphi(b)$ and $\varphi(ra) = r\varphi(a)$ for all $a, b \in A$ and $r \in \Z$. Observe that $\overline 0 = {\color{red} \overline n}$ so by hypothesis $\varphi(\overline 0) = \varphi(\overline n)$ but observe that $\varphi(\overline 0) = 0 \cdot a = 0$ and $\varphi(\overline n) = na$. Hence by equality $na = 0$. Conversely suppose that $na = 0$. We want to show that $\varphi : \Z/n\Z \to A$ defined by $\varphi(\overline k) = ka$ is a well-defined $R$-module homomorphism. Say $\overline k = \overline m$ then by property of cosets $k - m \in \Z/n\Z$ and so by definition $n \mid k - m$ and hence there exists $t \in \Z$ such that $k - m = nt$. Observe that
		\begin{align*}
			k - m &= nt\\
			(k - m)a &= nta\\
			ka - ma &= (na)t\\
			ka - ma &= 0\\
			ka &= ma 
		\end{align*}
		Thus we have $\varphi(\overline k) = \varphi(\overline m)$ and we can conclude that $\varphi$ is a well-defined $R$-module homomorphism.\\

		Now we want to show that $\operatorname{Hom}_\Z(\Z/n\Z, A) \cong A_n = \{a \in A : na = 0\}$. \textit{Note}: We are making the assumption that we want to show this in an isomorphism of $R$-modules as exercise does not specifiy group, ring or module isomorphism. Define $\Phi : A_n \to \operatorname{Hom}_\Z(\Z/n\Z, A)$ by $\Phi(a) = \varphi_a$. We will show that this is an $R$-module homomorphism, then that is a bijection.\\

		Let $a, a' \in A_n$, $r \in \Z$. Observe that
		\begin{align*}
			\Phi(a + a')(\overline k)&= \varphi_{a + a'}(\overline k)\\
			&= (a + a')k\\
			&= ak + a'k\\
			&= \varphi_a(\overline k) + \varphi_{a'}(\overline k)\\
			&= \Phi(a)(\overline k) + \Phi(a')(\overline k)
		\end{align*}
		So $\Phi(a + a') = \Phi(a) + \Phi(a')$ by definition. Moreover
		\begin{align*}
			\Phi(ra)(\overline k) &= \varphi_{ra}(\overline k)\\
			&= rak\\
			&= r\varphi_{a}(\overline k)\\
			&= r\Phi(a)(\overline k)
		\end{align*}
		Hence $\Phi(ra) = r\Phi(a)$ by definition. Conclude that $\Phi$ is an $R$-module homomorphism.\\

		Recall that $\ker \Phi = \{a \in A_n : \Phi(a) = 0\}$ and observe that
		\begin{align*}
			\ker \Phi &= \{a \in A_n : \Phi(a) = 0\}\\
			&= \{a \in A_n : \varphi_a(\overline k) = 0 \text{ for all k } \in \Z/n\Z\}\\
			&= \{0\}
		\end{align*}
		So we conclude that $\ker \Phi = \{0\}$ and hence $\Phi$ is injective.\\

		Let $\varphi \in \operatorname{Hom}_\Z(\Z/n\Z, A)$. Define $a = \varphi(\overline 1)$ and hence $na = n\varphi(\overline 1) = \varphi(n \overline 1) = \varphi(\overline n) = \varphi(\overline 0) = 0$ and hence $a \in A_n$. Observe that for all $\overline k \in \Z/n\Z$ it follows that $\varphi(\overline k) = \varphi(\overline k \cdot \overline 1) = \varphi(k \overline 1) = k \varphi(\overline 1) = ka = \varphi_a(\overline k)$. Thus by definition $\varphi = \varphi_a$. So for any $\varphi \in \operatorname{Hom}(\Z/n\Z, A_n)$ we can find $a \in A_n$ such that $\Phi(a) = \varphi_a = \varphi$. Conclude by definition that $\Phi$ is surjective.
	\end{proof}


\textbf{\S 10.2 \#5}: Exhibit all $\Z$-module homomorphisms from $\Z/30\Z$ to $\Z/21\Z$.
	\begin{proof}
	By previous exercise we know $\operatorname{Hom}(\Z/30\Z, \Z/21\Z) \cong (\Z/21\Z)_{30}$ where $(\Z/21\Z)_{30} = \{a \in \Z/21\Z : 30a = 0\} = A_{30}  = \{0, 7, 14\}$ and it has three elements. Hence we know that there are three homomorphisms from $\Z/30\Z$ to $\Z/21\Z$. The three homomorphisms are the ones defined by the trivial homomorphism, $\varphi_7(\overline x) = 7x$, $\varphi_{14}(\overline x) = 14x$.
	\end{proof}

\textbf{\S 10.2 \#6}: Prove that $\operatorname{Hom}_\Z(\Z/n\Z, \Z/m\Z) \cong \Z/(n, m)\Z$.
	\begin{proof}
		By previous exercise we know $\operatorname{Hom}_\Z(\Z/n\Z, \Z/m\Z) \cong (\Z/m\Z)_n = \{a \in \Z : na \equiv 0 \pmod m \}$. It will suffice to show that $(\Z/m\Z)_n \cong \Z/(n, m)\Z$. Let $d = \gcd(n, m)$, so by definition there exist $a, b$ relatively prime, such that $n = ad$ and $m = bd$. Observe that $b \in (\Z/m\Z)_n$ since
		\begin{align*}
			nb &\equiv (ad)b \pmod m\\
			&\equiv a(db) \pmod m\\
			&\equiv am \pmod m\\
			&\equiv 0 \pmod m
		\end{align*}
		\textit{Define $\varphi : \Z \to (\Z/m\Z)_n$ by $\varphi(z) = zb \pmod m$}. ({\color{green} Question: You said this was not well-defined with the given codomain, we edited the codomain to remove a redundancy, is this correct now?}) We must first show that this is a $\Z$-module homomorphism. Let $z, y \in \Z$, $r \in \Z$ and observe that
		\begin{align*}
			\varphi(z + y) &\equiv (z + y)b \pmod m\\
			&\equiv (zb + yb) \pmod m\\
			&\equiv zb \pmod m + yb \pmod m\\
			&\equiv \varphi(z) + \varphi(y)
		\end{align*}
		and
		\begin{align*}
			\varphi(rz) &\equiv (rz)b \pmod m\\
			&\equiv r(zb)\pmod m\\
			&\equiv r \varphi(z)\\
		\end{align*}

		We must now show that $\varphi$ is surjective. Choose $t \in (\Z/m\Z)_n$, by definition $nt \equiv 0 \pmod m$ and hence $m \mid nt$ or equivalently $bd \mid adt$ or equivalently $b \mid at$. Since $\gcd(a, b) = 1$, it must follow that $b \mid t$ so there exists $s \in \Z$ such that $t = sb$. Hence $\varphi(s) = sb \pmod m = t \pmod m$. Thus $\varphi$ is surjective. \\

		We will now show that $\ker \varphi = d\Z$. Observe that $\varphi(d) = db \pmod m \equiv m \pmod m \equiv 0 \pmod m$. So $d \in \ker \varphi$ and immediately $d\Z \subseteq \ker \varphi$. Now let $s \in \ker \varphi$. Then by definition $\varphi(s) = sb \pmod m \equiv 0 \pmod m$ so, by definition, $m \mid sb$ or equivalently $bd \mid sb$ or equivalently $d \mid s$ so $s \in d\Z$. Hence $\ker\varphi \subset d\Z$. Conclude that $\ker\varphi = d\Z$.\\

		By the First Isomorphism Theorem for Modules we have $\Z/\ker\varphi \cong (\Z/m\Z)_n$. Result follows by equality  $$\Z/\ker \varphi = \Z/d\Z = \Z/(n, m)\Z \cong (\Z/m\Z)_n \cong \operatorname{Hom}_\Z(\Z/n\Z, \Z/m\Z).\qedhere$$

	\end{proof}


\textbf{\S 10.2 \#7}: Let $z$ be a fixed element of the center of $R$. Prove that the map $m \to zm$ is an $R$-module homomorphism from $M$ to itself. Show that for a commutative ring $R$ the map from $R$ to $\operatorname{End}_R{M}$ given by $r \to rI$ is a ring homomorphism (where $I$ is the identity endomorphism).)
	\begin{proof}
		 Let $z$ be in the center of $R$. Define $\varphi : M \to M$ by $\varphi(m) = zm$. Let $m, m' \in M$, $r \in R$. Observe that $\varphi(m + m') = z(m + m') = zm + zm' = \varphi(m) + \varphi(m')$ and $\varphi(rm) = zrm = rzm = r\varphi(m)$. Thus $\varphi$ is an $R$-module homomorphism.\\

		Now let $R$ be a commutative ring and define $\Phi : R \to \operatorname{End}_R(M)$ by $\Phi(r) = rI$ where $I : M \to M$, defined by $I(m) = m$, is the identity endomorphism. We want to show that $\Phi$ is a ring homomorphism. Let $r, s \in R$ and observe that for all $m \in M$
		\begin{align*}
			\Phi(r + s)(m) &= (r + s)I(m)\\
			&= (r + s)m\\
			&= rm + sm\\
			&= rI(m) + sI(m)\\
			&= \Phi(r)(m) + \Phi(s)(m)
		\end{align*}
		So by definition $\Phi(r + s) = \Phi(r) + \Phi(s)$. Moreover,
		\begin{align*}
			\Phi(rs) &= rsI(m)\\
			&= rsI(m)I(m)\\
			&= rI(m) \cdot sI(m)\\
			&= \Phi(r)(m) \Phi(s)(m)
		\end{align*}
		And hence $\Phi(rs) = \Phi(r)\Phi(s)$ and we can conclude by definition that $\Phi$ is a ring homomorphism.
	\end{proof}


\begin{exercise}
\textbf{\S 10.2 \#8}: Let $\varphi : M \to N$ be an $R$-module homomorphism. Prove that $\varphi(\operatorname{Tor}(M)) \subseteq \operatorname{Tor}(N)$.
	\begin{proof}
		Recall that $\operatorname{Tor}(M) = \{m \in M : rm = 0$ for some $0 \neq r \in R\}$. Now it follows that $\varphi(\operatorname{Tor}(M)) = \{n \in N : n = \varphi(m)$ for some $m \in \operatorname{Tor}(M)\}$. Let $n \in \varphi(\operatorname{Tor}(M))$ then $n = \varphi(m)$ for some $m \in \operatorname{Tor}(M)$ by definition. Since $m \in \operatorname{Tor}(M)$ there exists $0 \neq r \in R$ such that $rm = 0$. Hence $rn = r \varphi(m) = \varphi(rm) = \varphi(0) = 0$. Conclude that $n \in \operatorname{Tor}(N)$ and hence $\varphi(\operatorname{Tor}(M)) \subseteq \operatorname{Tor}(N)$.
	\end{proof}


\textbf{\S 10.2 \#9}: Let $R$ be a commutative ring. Prove that $\operatorname{Hom}_R(R, M)$ and $M$ are isomorphic as left $R$-modules.
	\begin{proof}
		Define $\Phi : \operatorname{Hom}_R(R, M) \to M$ by $\Phi(\varphi) = \varphi(1)$. We must first show that this is an $R$-module homomorphism. Observe that for all $\varphi, \xi \in \operatorname{Hom}_R(R, M)$ and all $r \in R$ it follows that
		\begin{align*}
			\Phi(\varphi + \xi) &= (\varphi + \xi)(1)\\
			&= \varphi(1) + \xi(1)\\
			&= \Phi(\varphi) + \Phi(\xi) 
		\end{align*}
		and also by Proposition 2 we have
		\begin{align*}
			\Phi(r\varphi) &= (r\varphi)(1)\\
			&= r \varphi(1)\\
			&= r \Phi(\varphi).
		\end{align*}
		
		We must now show that $\Phi$ is injective. Suppose that $\Phi(\varphi) = \Phi(\xi)$. Then by definition $\varphi(1) = \xi(1)$ or equivalently $\varphi(1) - \xi(1) = 0$ and hence $(\varphi - \xi)(1) = 0$. But since $\varphi - \xi \in \operatorname{Hom}_R(R, M)$ it is an $R$-module homomorphism so $(\varphi - \xi)(x) = (\varphi - \xi)(x \cdot 1) = x(\varphi - \xi)(1) = x \cdot 0 = 0$ for all $x \in R$. Conclude that $\varphi(x) = \xi(x)$ for all $x \in R$ and hence by definition $\varphi = \xi$. Hence $\Phi$ is injective.\\

		We now show that $\Phi$ is surjective. Let $m \in M$ be arbitrary. We want to show that there exists $\varphi \in \operatorname{Hom}_R(R, M)$ such that $\Phi(\varphi) = m$. Define $\varphi : R \to M$ by $\varphi(x) = xm$. We need to show that $\varphi \in \operatorname{Hom}_R(R, M)$. Observe that $\varphi(x + y) = (x + y)m = xm + ym = \varphi(x) + \varphi(y)$ for all $x, y \in R$ and $\varphi(rx) = rxm = r \varphi(x)$ for all $x \in R$, $r \in R$. Hence we have shown that $\varphi$ is an $R$-module homomorphism. Now observe that $\Phi(\varphi) = \varphi(1) = 1 \cdot m = m$. Conclude by definition that $\Phi$ is surjective.\\

	Whence $\Phi$ is bijective and $\operatorname{Hom}_R(R, M) \cong M$.	
	\end{proof}


\textbf{\S 10.2 \#10}: Let $R$ be a commutative ring. Prove that $\operatorname{Hom}_R(R, R)$ and $R$ are isomorphic as rings.
	\begin{proof}
		Define $\Phi : \operatorname{Hom}_R(R, R) \to R$ by $\Phi(\varphi) = \varphi(1)$. By the previous exercise, all that remains to show is $\Phi(\varphi \circ \xi) = \Phi(\varphi) \Phi(\xi)$:
		\begin{align*}
			\Phi(\varphi \circ \xi) &= (\varphi \circ \xi)(1)\\
			&= \varphi(\xi(1))\\
			&= \varphi(\xi(1) \cdot 1)\\
			&= \xi(1) \varphi(1)\\
			&= \varphi(1) \xi(1)\\
			&= \Phi(\varphi) \Phi(\xi).
		\end{align*}
	\end{proof}


\textbf{\S 10.2 \#11}: Let $A_1, A_2, \ldots, A_n$ be $R$-modules and let $B_i$ be submodules of $A_i$ for each $i = 1, 2, \ldots, n$. Prove that $$(A_1 \times A_2 \times \cdots \times A_n)/(B_1 \times B_2 \times \cdots \times B_n) \cong (A_1/B_1) \times (A_2/B_2) \times \cdots \times (A_n/B_n).$$
	\begin{proof}
		Define $\varphi : A_1 \times A_2 \times \cdots \times A_n \to (A_1/B_1) \times (A_2/B_2) \times \cdots \times (A_n/B_n)$ by $\varphi(a_1, a_2, \ldots, a_n) = (a_1 + B_1, a_2 + B_2, \ldots, a_n + B_n)$. \\
		
		
		Let $x, y \in A_1 \times A_2 \times \cdots \times A_n$ where $x = (a_1, a_2, \ldots, a_n)$ and $y = (a_1', a_2', \ldots, a_n')$. Observe that
		\begin{align*}
			\varphi(x + y) &= \varphi((a_1, a_2, \ldots, a_n) + (a_1', a_2', \ldots, a_n'))\\
			&= \varphi(a_1 + a_1', a_2 + a_2', \ldots, a_n + a_n')\\
			&= (a_1 + a_1' + B_1, a_2 + a_2' + B_2, \ldots, a_n + a_n' + B_n)\\
			&= (a_1 + B_1 + a_1' + B_1, a_2 + B_2 + a_2' + B_2, \ldots, a_n + B_n + a_n' + B_n)\\
			&= (a_1 + B_1, a_2 + B_2, \ldots, a_n + B_n) + (a_1' + B_1, a_2' + B_2, \ldots, a_n' + B_n)\\
			&= \varphi(a_1, a_2, \ldots, a_n) + \varphi(a_1', a_2', \ldots, a_n')\\
			&= \varphi(x) + \varphi(y)
		\end{align*}
		and for $r \in R$
		\begin{align*}
			\varphi(rx) &= \varphi(r(a_1, a_2, \ldots, a_n))\\
			&= \varphi(ra_1, ra_2, \ldots, ra_n)\\
			&= (ra_1 + B_1, ra_2 + B_2, \ldots, ra_n + B_n)\\
			&= (r(a_1 + B_1), r(a_2 + B_2), \ldots, r(a_n + B_n))\\
			&= r(a_1 + B_1, a_2 + B_2, \ldots, a_n + B_n)\\
			&= r\varphi(a_1, a_2, \ldots, a_n)\\
			&= r\varphi(x)
		\end{align*}
		Thus $\varphi$ is an $R$-module homomorphism.\\

		Now we want to show that $\ker \varphi = B_1 \times B_2 \times \cdots \times B_n$. Observe that
		\begin{align*}
			\ker \varphi &= \{x \in A_1 \times A_2 \times \cdots \times A_n : \varphi(x)= 0\}\\
			&= \{(a_1, a_2, \ldots, a_n) \in A_1 \times A_2 \times \cdots \times A_n : \varphi(a_1, a_2, \ldots, a_n) = 0\}\\
			&= \{(a_1, a_2, \ldots, a_n) \in A_1 \times A_2 \times \cdots \times A_n : (a_1 + B_1, a_2 + B_2, \ldots, a_n + B_n) = (0, 0, \ldots, 0)\}\\
			&= \{(a_1, a_2, \ldots, a_n) \in A_1 \times A_2 \times \cdots \times A_n : a_1 \in B_1, a_2 \in B_2, \ldots, a_n \in B_n\}
		\end{align*}
		Hence $\ker\varphi \subseteq B_1 \times B_2 \times \cdots \times B_n$ and trivially $B_1 \times B_2 \times \cdots \times B_n \subseteq \ker \varphi$ by construction of $\varphi$. Conclude that $\ker \varphi = B_1 \times B_2 \times \cdots \times B_n$.\\

		The mapping is trivially surjective. Applying first Isomorphism theorem yields the result.
	\end{proof}


\textbf{\S 10.3 \#3}: Show that the $F[x]$-modules in Exercises 18 and 19 of Section 1 are both cyclic.
	\begin{proof}
		For Problem 18: $F = \R$, $V = \R^2$, $T : V \to V$ is defined by $T(x, y) = (y, -x)$. Let $(a, b) \in \R^2$ be arbitrary. Observe that $(ax + b)(0, 1) = aT(0, 1) + b(0, 1) = a(1, 0) + b(0, 1) = (a, b)$. Hence it follows by definition that $V = \R[x](0, 1)$. Moreover it can also be written as $V = \R[x](1, 0)$ with $p(x) = a - bx$, so the representation is not unique.\\

		For Problem 19: $F = \R$, $V = \R^2$, $T : V \to V$ is defined by $T(x, y) = (0, y)$. Let $(a, b) \in \R^2$ be arbitrary. Observe that $(a + (b - a)x)(1, 1) = (a, a) + (b - a)T(1, 1) = (a, a) + (0, b - a) = (a, b)$. Hence it follows by definition that $V = \R[x](1, 1)$.
	\end{proof}
\end{exercise}

\textbf{\S 10.3 \#4}: An $R$-module $M$ is called a \textit{torsion} module if for each $m \in M$ there is a nonzero element $r \in R$ such that $rm = 0$, where $r$ may depend on $m$ (i.e., $M = \operatorname{Tor}(M)$ in the notation of Exercise 8 of Section 1). Prove that every finite abelian group is a torsion $\Z$-module. Give an example of an infinite abelian group that is a torsion $\Z$-module.
	\begin{proof}
	
		Let $M$ be a finite abelian group. Let $m \in M$. We want to show that there exists $0 \neq r \in R = \Z$ such that $rm = 0$. Consider $1m, 2m, 3m, 4m, \ldots$. These are not all distinct, because if they were we would have infinitely many, a contradiction. So we are assured $km = lm$ for some $k, l \in \Z$ nonzero with $k \neq l$. It follows that $km - lm = 0$ and hence by property of modules, $(k - l)m = 0$. Finally observe that $k - l \neq 0$ so $m \in \operatorname{Tor}(M)$. Conclude that $M = \operatorname{Tor}(M)$.\\

		As for the example. Let $n \in \Z$ be greater than 1. Consider $A = \Z/n\Z \times \Z/n\Z \times \cdots$ and observe that this is an infinite abelian group. This can be seen as a $\Z$-module. Let $(a_1, a_2, \ldots) \in \Z/n\Z \times \Z/n\Z \times \cdots$ be arbitrary. Observe that $n(a_1, a_2, \ldots) = (na_1, na_2, \ldots) = (0, 0, \ldots) = \textbf 0$.
	\end{proof}
\textbf{\S 10.3 \#5}: Let $R$ be an integral domain. Prove that every finitely generated torsion $R$-module has a nonzero annihilator i.e., there is a nonzero element $r \in R$ such that $rm = 0$ for all $m \in M$ -- here $r$ does not depend on $m$ (the annihilator of a module was defined in Exercise 9 of Section 1). Give an example of a torsion $R$-module whose annihilator is the zero ideal.
	\begin{proof}
		Let $M$ be a finitely generated torsion $R$-module where $R$ is an integral domain. By definition $M = Rm_1 + \cdots + Rm_n$ for some $m_1, m_2, \ldots, m_n \in M$. Let $m \in M$. Since $M$ is finitely generated there exist $r_1, r_2, \ldots, r_n \in R$ such that $m = r_1m_1 + \cdots + r_nm_n$. Since $M$ is torsion, there exists $0 \neq \overline{r_i} \in R$ such that $\overline{r_i}m_i = 0$ for $i = 1, 2, \ldots, n$. Define $r = \overline{r_1} \overline{r_2 \cdots r_n}$. Note that $R$ is an integral domain, so that $r \neq 0$. Now observe that {\color{red} $$rm_i = (\overline{r_1r_2 \cdots r_{i - 1}r_{i + 1} \cdots {r_n}})(\overline{r_i}m_i) =  (\overline{r_1r_2 \cdots r_{i - 1}r_{i + 1} \cdots {r_n}}) \cdot 0 = 0 \text{ for all } i.$$ Thus since $rm = \D \sum_{i = 1}^n r_i(rm_i) = \sum_{i = 1}^n r_i \cdot 0 = 0$}

%		\begin{align*}
%			rm &= r(r_1m_1 + \cdots + r_nm_n)\\
%			&= (r_1 \overline{r_2} \cdots \overline{r_n})\overline{r_1}m_1 + \cdots + (\overline{r_1} \cdots \overline{r_{n-1}} r_n) \overline{r_n}m_n\\
%			&= (r_1 \overline{r_2} \cdots \overline{r_n}) \cdot 0 + \cdots + (\overline{r_1} \cdots \overline{r_{n-1}} r_n) \cdot 0\\
%			&= 0
%		\end{align*}
		hence it follows that $0 \neq r \in \operatorname{Ann}(m)$ and thus $\operatorname{Ann}(m) \neq 0$.\\

		As for the example: recall that $\Q$ is not finitely generated {\color{red} over $\Z$ for suppose by way of contradiction that it was finitely generated, then there would exist a basis $x_1, \ldots, x_n \in \Q$ where $\D x_i = \frac{a_i}{b_i}$ for $i = 1, \ldots, n$ with $\gcd(a_i, b_i) = 1$. Choose $p > \D \max_{1 \leq i \leq m} \vert b_i \vert$ be a prime; then by hypothesis $\frac1p = r_1x_1 + \cdots + r_nx_n$ for some $r_i \in \Z$. Multiplying both sides by $pb_1\cdots b_n$ we get $b_1b_2 \cdots b_n = pq$ for some integer $q$. Since $p$ is prime, $p \mid b_i$ for some $i = 1, 2, \ldots, n$ a contradiction.} Hence $M = \Q/\Z$ is also not finitely generated {\color{red} over $\Z$ since if we suppose by way of contradiction that $\Q/\Z$ is finitely generated then it has a basis $\overline x_1, \ldots, \overline x_n \in \Q/\Z$ with $\overline x_i = x_i + \Z$. In particular, for any $y \in \Q$, we can consider $\overline y$. Observe that
		\begin{align*}
			y + \Z &= \overline y\\
			&= r_1\overline x_1 + \cdots + r_n \overline x_n\\
			&= (r_1x_1 + \cdots + r_nx_n) + \Z
		\end{align*}
		for some $r_i \in \Z$ and in particular $y - (r_1x_1 + \cdots + r_nx_n) \in \Z$ so there exists $z \in \Z$ such that $y =  (r_1x_1 + \cdots + r_nx_n) + z \cdot 1$. Hence we have just shown that $\Q$ is finitely generated, a contradiction to our previous result. Conclude that $\Q/\Z$ is not finitely generated.}\\ 

Observe that $M$ is torsion since for any rational, non-integer number $x$, multiplication by its denominator, which \textbf{is} an integer, yields an integer, which in this case would be 0 in $M$. Suppose by way of contradiction that there is a nonzero annihilator, say $0 \neq a \in R = \Z$. Choose $b \in \Z$ such that $b \nmid a$. Now $a \cdot 1/b = 0$ by property of being annihilator so $a/b = k$ is an integer. But then $a = bk$ and hence $b \mid a$, a contradiction. So there are no nonzero annihilators.
	\end{proof}


\textbf{\S 10.3 \#9}: An $R$-module $M$ is called \textit{irreducible} if $M \neq 0$ and if $0$ and $M$ are the only submodules of $M$. Show that $M$ is irreducible if and only if $M \neq 0$ and $M$ is a cyclic module with any nonzero element as its generator. Determine all the irreducible $\Z$-modules.
	\begin{proof}
		Suppose $M$ is irreducible. By definition $M \neq 0$. Let $0 \neq m \in M$. Note that $Rm \subseteq M$ is nonzero since $1m = m \in Rm$. Since $M$ is irreducible and we have already shown $Rm \neq 0$, conclude that $Rm = M$ for any $0 \neq m \in M$.\\

		Conversely suppose that $M \neq 0$ and $M$ is a cyclic module with any nonzero element as generator. Let $N$ be a submodule of $M$, so $0 \subseteq N \subseteq M$. If $N = 0$ we are done, so suppose $N \neq 0$, then there exists $n \in N$ such that $n \neq 0$. But $n \in M$ and so $M = Rn$. Since $Rn \subseteq N$ immediately we have just shown $M \subseteq N$. Since both $N \subseteq M$ and $M \subseteq N$ we have $M = N$.\\

		Let $M$ be an irreducible $\Z$-module then $M$ is {\color{red} a cyclic module and since $\Z$-modules are just abelian groups, we are looking at cyclic groups that are abelian, i.e. all cyclic groups. So either $M \cong \Z$ or $M \cong \Z/n\Z$ for some $n \in \Z$. If $M \cong \Z$ then $M$ has infinitely many subgroups and hence infinitely many submodules, a contradiction to being irreducible. Conclude that $M \cong \Z/n\Z$ for some $n \in \Z$. Since M can only have two subgroups, conclude $M \cong \Z/p\Z$ for some prime $p$.}
	\end{proof}


\textbf{\S 10.3 \#10}: Assume $R$ is commutative. Show that an $R$-module $M$ is irreducible if and only if $M$ is isomorphic (as an $R$-module) to $R/I$ where $I$ is a maximal ideal of $R$. [By the previous exercise, if $M$ is irreducible there is a natural map $R \to M$ defined by $r \mapsto rm$, where $m$ is any fixed nonzero element of $M$.]
	\begin{proof}
		Let $M$ be an $R$-module. Suppose $M$ is irreducible. Then by definition $M \neq 0$ and the only submodules of $M$ are $0$ and $M$. Define $\varphi_m : R \to M$ by $\varphi(r) = rm$ where $0 \neq m$ is a fixed element of $M$. We want to show that this is an $R$-module homomorphism. Let $x, y \in R$ and $r \in R$ and observe that $\varphi(x + y) = (x + y)m = xm + ym = \varphi(x) + \varphi(y)$ and $\varphi(rx) = rxm = r\varphi(x)$. This mapping is surjective since $M$ is a cyclic module with any nonzero element as its generator. So by the First Isomorphism Theorem for Modules, it follows that $R/\ker \varphi_m \cong M$. $\ker \varphi_m$ is a submodule by a previous exercise and is trivially an ideal of $R$. It will suffice to show that $\ker \varphi_m$ is a maximal ideal for the result to follow. Let $0 \neq \overline r \in R/\ker \varphi_m$ where $\overline r = r + \ker \varphi_m$. It follows that $\varphi_m(r) = rm \neq 0$. We want to show that $\overline r$ has a multiplicative inverse. Since $M$ is irreducible and $rm \neq 0$, we have $M = R(rm)$ by previous exercise. In particular $m = s(rm)$ for some $s \in R$. Then by equality $1\cdot m - (sr)m = 0$ resulting in $(1 - sr) \in \ker \varphi_m$, By property of cosets, we have $1 + \ker \varphi_m = sr + \ker \varphi_m = (s + \ker \varphi_m)(r + \ker \varphi_m)$. We have just shown that $\overline 1 = \overline s\overline r$ giving us $\overline s$ as the multiplicative inverse of $\overline r$, an arbitrary nonzero element (by commutativity of $R$, it is both a left and right inverse). Thus $R/\ker \varphi_m$ is a field and so $\ker \varphi_m$ is a maximal ideal.\\

		Conversely suppose $M \cong R/I$ where $I$ is a maximal ideal of $R$ (as an $R$-module homomorphism). Then by definition there exists $\varphi : M \to R/I$ such that $\varphi(m + n) = \varphi(m) + \varphi(n)$ and $\varphi(rm) = r \varphi(m)$ for all $m, n \in M$, $r \in R$. Observe that $M \neq 0$ since $M \cong R/I$ and $I$ is maximal, by definition $I \neq R$ so $R/I$ cannot be trivially 0 so $M$ cannot be trivially 0. Suppose $N$ is a submodule of $M$, then $0 \subseteq N \subseteq M$. Suppose, by way of contradiction, that $0 \neq N \neq M$. Note that $\varphi(N)$ is a submodule of $R/I$ and trivially $\varphi(N)$ is an ideal of $R/I$ which is not 0 and not $R/I$, a contradiction to $0 \neq N \neq M$ since $R/I$ is a field, the only ideals of $R/I$ are 0 and $R/I$. So either $0 = N$ or $N = M$. Conclude by definition that $M$ is irreducible.
	\end{proof}


\textbf{\S 10.3 \#15}: An element $e \in R$ is called a \textit{central idempotent} if $e^2 = e$ and $er = re$ for all $r \in R$. If $e$ is a central idempotent in $R$, prove that $M = eM \oplus (1 - e)M$. [Recall Exercise 14 in Section 1.]
	\begin{proof}
		Suppose $r$ is a central idempotent in $R$. We want to show that $M = eM \oplus (1 - e)M$, that is, any $m \in M$ can be written \textit{uniquely} of the form $em_1 + (1 - e)m_2$ for some $m_1, m_2 \in M$. Let $m \in M$ and observe that $m = em + (m - em)$ so $M \subseteq eM + (1 - e)M$ and $eM + (1 - e)M \subseteq M$ trivially by closure of modules. Now we need to show the uniqueness. Suppose $m \in eM \cap (1 - e)M$ then $m = em_1 = (1 - e)m_2$ for some $m_1, m_2 \in M$. But then $em_1 = m_2 - em_2$ or equivalently $e(m_1 + m_2) = m_2$. Multiplying both sides by $e$, we get $e^2(m_1 + m_2) = em_2$ and since $e^2 = e$ we have $em_1 + em_2 = em_2$, making $em_1 = 0$. Hence $m = 0$. Conclude that $eM \cap (1 - e)M = 0$.
	\end{proof}


\textbf{\S 10.3 \#16}: For any ideal $I$ of $R$ let $IM$ be the submodule defined in Exercise 5 of Section 1. Let $A_1, \ldots, A_k$ be any ideals in the ring $R$. Prove that the map $$M \to M/A_1M \times \cdots \times M/A_kM \text{ defined by } m \mapsto (m + A_1M, \ldots, m + A_kM)$$ is an $R$-module homomorphism with kernel $A_1M \cap A_2M \cap \cdots \cap A_kM$.
	\begin{proof}
		We first must show that $\varphi : M \to M/A_1M \times \cdots \times M/A_kM$ defined by $\varphi(m) = (m + A_1M, \ldots, m + A_kM)$ is an $R$-module homomorphism. Let $m_1, m_2 \in M$, $r \in R$. Observe that
		\begin{align*}
			\varphi(m_1 + m_2) &= (m_1 + m_2 + A_1M, \ldots, m_1 + m_2 + A_kM)\\
			&= \big((m_1 + A_1M) + (m_2 + A_1M), \ldots, (m_1 + A_kM) + (m_2 + A_kM)\big)\\
			&= (m_1 + A_1M, \ldots, m_1 + A_kM) + (m_2 + A_1M, \ldots, m_2 + A_kM)\\
			&= \varphi(m_1) + \varphi(m_2)
		\end{align*}
		and
		\begin{align*}
			\varphi(rm_1) &= (rm_1 + A_1M, \ldots, rm_2 + A_kM)\\
			&= \big(r(m_1 + A_1M), \ldots, r(m_1 + A_kM)\big)\\
			&= r(m_1 + A_1M, \ldots, m_1 + A_kM)\\
			&= r\varphi(m_1)
		\end{align*}
		
		Observe
		\begin{align*}
			\ker \varphi &= \{m \in M : \varphi(m) = 0\}\\
			&= \{m \in M : (m + A_1M, \ldots, m + A_kM) = (0, \ldots, 0)\}\\
			&= \{m \in M : m \in A_1M, \ldots, m \in A_kM\}\\
			&= A_1M \cap \cdots \cap A_kM \qedhere
		\end{align*}
	\end{proof}


\textbf{\S 10.3 \#22}: Let $R$ be a Principal Ideal Domain, let $M$ be a torsion $R$-module (cf. Exercise 4) and let $p$ be a prime in $R$ (do not assume $M$ is finitely generated, hence it need not have a nonzero annihilator -- cd. Exercise 5). The \textit{$p$-primary component} of $M$ is the set of all elements of $M$ that are annihilated by some positive power of $p$.
	\begin{enumerate}
		\item Prove that the $p$-primary component is a submodule. [See Exercise 13 in Section 1.]
			\begin{proof}
				Let $A$ be the $p$-primary component, i.e.,  $A = \{m \in M : p^im = 0$ for some $i \in \N\}$. Observe that $pm = 0$ for $m = 0$ so $A \neq \emptyset$. Let $m, n \in A$, $r \in R$ a PID. Since $m \in A$ there exists $i \in \N$ such that $p^im = 0$. Since $n \in A$ there exists $j \in \N$ such that $p^jn = 0$. Choose $l = \max\{i, j\}$ and observe that
				\begin{align*}
					p^l(m + rn) &= p^lm + rp^ln\\
					&= p^{l - i}(p^im) + rp^{l - j}(p^jn)\\
					&= p^{l - i} \cdot 0 + rp^{l - j} \cdot 0\\
					&= 0
				\end{align*}
				Conclude that $m + rn \in A$ and thus $A$ is a submodule.
			\end{proof}
		\item Prove that this definition of $p$-primary component agrees with the one given in Exercise 18 when $M$ has a nonzero annihilator.
			\begin{proof}
				Suppose $\operatorname{Ann}(M) \neq 0$, then there exists $0 \neq a \in \operatorname{Ann}(M)$ such that $\operatorname{Ann}(M) = (a)$ since $R$ is a Principal Ideal Domain. Since every PID is a UFD and primes are irreducibles in here, we can decompose $a$, say $a = p_1^{\alpha_1}\cdots p_n^{\alpha_n}$. We want to show that $M_{p_i} = \{m \in M : p_i^jm = 0$ for some $j \in \N\}$ is equal to $M_i = \{m \in M : p_i^{\alpha_i}m = 0\}$.\\

				Let $m \in M_i$, then by definition we know that $p_i^{\alpha_i}m = 0$ and immediately it follows that $m \in M_{p_i}$. Conclude that $M_i \subseteq M_{p_i}$.\\

				Conversely suppose that $m \in M_{p_i}$ then by definition there exists $j \in \N$ such that $p_i^jm = 0$. Consider $(a, p_i^j)$ and observe that since $R$ is a PID, it must follow that $(a, p_i^j) = (b)$ for some $b \in R$. But also note that $(p_i^j) \subseteq (a, p_i^j) = (b)$, so by property of ideals, we have $b \mid p_i^j$ which leads us to conclude by property of primes that $b = p_i^t$ for some $t \leq j$. But we also know that $(a) \subseteq (p_i^t)$ so it follows that
				\begin{align*}
					p_i^t &\mid p_1^{\alpha_1} \cdots p_{i - 1}^{\alpha_{i - 1}} p_i^{\alpha_i} p_{i + 1}^{\alpha_{i + 1}} \cdots p_n^{\alpha_n}\\
					p_i^{t - \alpha_i}p_i^{\alpha_i} &\mid  p_1^{\alpha_1} \cdots p_{i - 1}^{\alpha_{i - 1}} p_i^{\alpha_i} p_{i + 1}^{\alpha_{i + 1}} \cdots p_n^{\alpha_n}\\
					p_i^{t - \alpha_i} &\mid  p_1^{\alpha_1} \cdots p_{i - 1}^{\alpha_{i - 1}} p_{i + 1}^{\alpha_{i + 1}} \cdots p_n^{\alpha_n}
				\end{align*}
				So by definition $  p_1^{\alpha_1} \cdots p_{i - 1}^{\alpha_{i - 1}} p_{i + 1}^{\alpha_{i + 1}} \cdots p_n^{\alpha_n} = p_i^{t - \alpha_i}s$ for some $s \in R$. It must follow that $t - \alpha_i = 0$, or equivalently $t = \alpha_i$, because if it does not, we have a contradiction, since $p_i$ is not in $\{p_1, \ldots, p_{i-1}, p_{i + 1}, \ldots, p_n\}$. So since $(p_i^j) \subseteq (a, p_i^j) =  (p_i^{\alpha_i})$ we have $p_i^{\alpha_i} = p_i^j \cdot k$ for some $k \in R$. So $p_i^{\alpha_i}m = p_i^j \cdot k \cdot m = k \cdot (p_i^j \cdot m) = k \cdot 0 = 0$. Conclude that $m \in M_i$ and hence $M_{p_i} \subseteq M_i$.\\

				Finally, by double inclusion, we can conclude that $M_i = M_{p_i}$.
			\end{proof}
		\item Prove that $M$ is the (possibly infinite) direct sum of its $p$-primary components, as $p$ runs over all primes of $R$.
		\begin{proof}
		
		Denote $P \subseteq R$ as the set of primes in $R$. Let $m \in M$. Since $M$ is a torsion $R$-module, there exist $r \in R$ such that $rm =0$. $R$ is a PID, so can write $r$ into its unique prime factorization, say $r = \prod_1^n p_i^{\alpha_i}$. Define $q_j = \prod_{i\neq j} p_i^{\alpha_i}$. Then $(q_1,q_2,...,q_n) = R$. Thus we can write $1 = \sum_1^n a_iq_i$ for some $a_i \in R$. We have $0 = rm = (p_i^{\alpha_i}q_i)m = p_i^{\alpha_i}(q_im)$. Thus $q_im \in M_{p_i}$, so that $a_iq_im \in M_{p_i}$. But $m = 1\cdot m = (\sum_1^n a_iq_i )\cdot m = \sum_1^n a_iq_im \in \sum_1^n M_{p_i}$. Thus $m \in \sum_{p \in P} M_p$ and we have $M = \sum_{p \in P} M_p$. \\
		
		Claim that this sum is a direct sum. First, fix a prime $p\in P$ and denote $Q = P \setminus \{p\}$. Suppose $m \in M_{p} \cap (\sum_Q M_q)$. Write $m = m_{p} = \sum_Qm_q$, where $m_{p} \in M_{p}$ and $m_q \in M_q$ for each $q \in Q$. By definition, we know there exist $k \geq 0$ such that $p^km = 0$. Thus $(p^k) \subseteq Ann_R(m)$. We also know there exist $e_q \geq 0$ such that $q^{e_q}m_q = 0$, for each $q \in Q$. Claim that $\prod_Q q^{e_q} \in Ann_R(m)$. To see this, observe that $\prod_Qq^{e_q}m = [\prod_Qq^{e_q}][\sum_Qm_q] = \sum_Q[(\prod_Qq^{e_q})m_q] = \sum_Q 0 = 0$. Now $(\prod_Q q^{e_q}) \subseteq Ann_R(m)$, implying $(p^k, \prod_Q q^{e_q}) \subseteq Ann_R(m)$. But $p^k$ and $\prod_Q q^{e_q}$ are relatively prime by construction, so $(p^k, \prod_Q q^{e_q}) = R$, forcing $Ann_R(m) = R$. In particular, we have $m = 1\cdot m = 0$. Thus $M_{p} \cap (\sum_Q M_q) = 0$ and we have that $m$ is written uniquely in $\sum_{p \in P} M_p$. I.e., $M = \bigoplus\sum_{p \in P} M_p$. 
		
		\end{proof}
		
	\end{enumerate}

\end{document}




