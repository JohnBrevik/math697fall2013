\documentclass[8pt]{amsart}
\usepackage{amsmath, amssymb}
\usepackage{amsfonts}
\usepackage{mathrsfs}
\usepackage[arrow,matrix,curve,cmtip,ps]{xy}
\usepackage{paralist}
\usepackage[left=1in,top=1in,right=1in,bottom=1in]{geometry}
\usepackage{amsthm}

\allowdisplaybreaks

\theoremstyle{plain}% default

\theoremstyle{definition}
\newtheorem{theorem}{Theorem}[section]
\newtheorem{lemma}{Lemma}[section]
\newtheorem*{proposition}{Proposition}
\newtheorem*{corollary}{Corollary}
\newtheorem*{KL}{Klein’s Lemma}

\newtheorem*{definition}{Definition}
\newtheorem{conjecture}{Conjecture}[section]
\newtheorem{example}{Example}[section]
\newtheorem*{exercise}{Exercise}%[section]
\newtheorem*{notation}{Notation}
\newtheorem*{remark}{Remark}

\theoremstyle{remark}
\newtheorem*{note}{Note}
\newtheorem{case}{Case}

%this has equations numbered within sections 1.1,1.2, ... 2.1,...
\numberwithin{equation}{section}

\makeatletter
\newenvironment{solution}
               {\let\oldqedsymbol=\qedsymbol%
                \def\@addpunct##1{}%
                \renewcommand{\qedsymbol}{$\blacktriangleleft$}%
                \begin{proof}[\itshape Solution.]}%
               {\end{proof}%
                \renewcommand{\qedsymbol}{\oldqedsymbol}}
\makeatother

\def\upint{\mathchoice%
    {\mkern13mu\overline{\vphantom{\intop}\mkern7mu}\mkern-20mu}%
    {\mkern7mu\overline{\vphantom{\intop}\mkern7mu}\mkern-14mu}%
    {\mkern7mu\overline{\vphantom{\intop}\mkern7mu}\mkern-14mu}%
    {\mkern7mu\overline{\vphantom{\intop}\mkern7mu}\mkern-14mu}%
  \int}
\def\lowint{\mkern3mu\underline{\vphantom{\intop}\mkern7mu}\mkern-10mu\int}


%-------------------------------------------
%       Begin Local Macros
%-------------------------------------------

\newcommand{\Z}{\mathbb{Z}}
\newcommand{\N}{\mathbb{N}}
\newcommand{\Q}{\mathbb{Q}}
\newcommand{\R}{\mathbb{R}}
\newcommand{\C}{\mathbb{C}}
\newcommand{\T}{\mathbb{T}}
\newcommand{\D}{\displaystyle}
\newcommand{\im}{\operatorname{im}}
\newcommand{\coker}{\operatorname{coker}}
\newcommand{\ind}{\operatorname{ind}}
\newcommand{\rank}{\operatorname{rank}}
\newcommand\mc[1]{\marginpar{\sloppy\protect\footnotesize #1}}

%-------------------------------------------
%       End Local Macros
%-------------------------------------------
\begin{document}
\title[MATH 697]{Introduction to Homological Algebra}

% \author{Robert Cardona}

\author{
	Robert Cardona %\textit{mrrobertcardona@gmail.com}
	\and
	Massy Khoshbin %\textit{massy255@gmail.com}
	\and
	Siavash Mortezavi %\textit{siavash.mortezavi@gmail.com}
}


\address{Department of Mathematics \\ California State University Long Beach}
\email{mrrobertcardona@gmail.com \and massy255@gmail.com \and siavash.mortezavi@gmail.com}

\date{\today}


\maketitle

%%%%%%%%%%%%%%%%%%%%%%%%%%%%%%%%%%%%%%%%%%%%%%%%
\setcounter{section}{-1}
\section{MATH 697 Homework Zero.Two}
%%%%%%%%%%%%%%%%%%%%%%%%%%%%%%%%%%%%%%%%%%%%%%%%


\textbf{ AM 2.1}: Show that $(\Z/m\Z) \otimes (\Z/n\Z) = 0$ if $m$ and $n$ are coprime.
	\begin{proof}
		Choose $a \otimes b \in \Z/m\Z \otimes \Z/n\Z$. Since $m$ and $n$ are coprime, there exist $s, t \in \Z$ such that $ms + nt = 1$ Observe that $$a = a \cdot 1 = a(ms + nt) = ams + ant \equiv ant \pmod m.$$ Now observe that $$a \otimes b = atn \otimes b = a \otimes nb = at \otimes 0 = 0.$$
		We have shown that any simple tensor is zero, so any finite linear combination of simple tensors is zero. Conclude $(\Z/m\Z) \otimes (\Z/n\Z) = 0$.\\
	\end{proof}

\textbf{ AM 2.2}: Let $R$ be a ring, $I$ an ideal of $R$, $M$ an $R$-module. Show that $(R/I) \otimes_RM$ is isomorphic to $M/IM$. 
	\begin{proof}
<<<<<<< HEAD
		Define $\varphi : R/I \times M \to M/IM$ by $\varphi(r + I, m) = rm + IM$, which we shall henceforth write as $\varphi(\overline{r}, m) = \overline{rm}$. Let $(\overline{r}, m) = (\overline{s}, m)$. Then $\overline{r} = \overline{s} \implies r \in \overline{s} \implies r=s+i$, some $i \in I$. Then $\varphi(\overline{r},m)=\overline{rm}=\overline{(s+i)m}=\overline{sm+im}=\overline{sm}+\overline{im}=\overline{sm}+\overline{0}=\overline{sm}=\varphi(\overline{s},m)$. Thus $\varphi$ is well-defined.\\ 
		
		Observe $\varphi(\overline{r}+\overline{s},m)=\varphi(\overline{r+s},m)=\overline{(r+s)m}=\overline{rm+sm}=\overline{rm}+\overline{sm}=\varphi(\overline{r},m)+\varphi(\overline{s},m)$. Similarly, $\varphi(\overline{r},m+n)=\varphi(\overline{r},m)+\varphi(\overline{r},n)$. Lastly, $\varphi(\overline{rs},m)=\overline{(rs)m}=\overline{r(sm)}=\varphi(\overline{r},sm)$. Thus $\varphi$ is $R$-biadditive (In fact, $\varphi$ is $R$-bilinear).\\
		
		 Now we are guaranteed a unique $R$-homomorphism $\phi : R/I \otimes_RM \rightarrow M/IM$ given by $\phi(\overline{r} \otimes m)=\overline{rm}$. Notice if we define $f : M/IM \rightarrow R/I \otimes_RM$ via $f(\overline{m})=\overline{1}\otimes m$ then $f$ is a $\Z$-homomorphism which is the inverse of $\phi$ (as a function). Thus $\phi$ is a bijective function and hence an isomorphism when considered as an $R$-map. 
		 \end{proof}

	
=======
		Define $\varphi : R/I \otimes_RM \to M/IM$ by $\varphi(r + I \otimes m) = rm + IM$. We must first show this mapping is well-defined. Suppose $r_1 + I \otimes m_1 = r_2 + I \otimes m_2$.\\

		\textbf{Question:} Can we assume $m_1 = m_2$? If we can then the well-definedness follows immediately. But I am unsure. Recall that the simple tensor is just a coset itself, i.e., $r + I \otimes m = (r + I, m) + (R/I) \otimes_RM$.
	\end{proof}

	\begin{proof} (\textit{Alternate}) Choose $a \otimes b \in \Z/m\Z \otimes \Z/n\Z$. Since $m$ and $n$ are coprime, there exist $s, t \in \Z$ such that $ms + nt = 1$ Observe that $$a = a \cdot 1 = a(ms + nt) = ams + ant \equiv ant \pmod m.$$ Now observe that $$a \otimes b = atn \otimes b = a \otimes nb = at \otimes 0 = 0. \qedhere $$

	\end{proof}
>>>>>>> ca9d6167759a2d99fd6be07302eb45c7fe70a897
\end{document}
