\documentclass[8pt]{amsart}
\usepackage{amsmath, amssymb}
\usepackage{amsfonts}
\usepackage{mathrsfs}
\usepackage[arrow,matrix,curve,cmtip,ps]{xy}
\usepackage{paralist}
\usepackage[left=1in,top=1in,right=1in,bottom=1in]{geometry}
\usepackage{amsthm}
\usepackage{tikz}
\usetikzlibrary{arrows,chains,matrix,positioning,scopes}

\allowdisplaybreaks

\theoremstyle{plain}% default

\theoremstyle{definition}
\newtheorem{theorem}{Theorem}[section]
\newtheorem{lemma}{Lemma}[section]
\newtheorem*{proposition}{Proposition}
\newtheorem*{corollary}{Corollary}
\newtheorem*{KL}{Klein’s Lemma}

\newtheorem*{definition}{Definition}
\newtheorem{conjecture}{Conjecture}[section]
\newtheorem{example}{Example}[section]
\newtheorem*{exercise}{Exercise}%[section]
\newtheorem*{notation}{Notation}
\newtheorem*{remark}{Remark}

\theoremstyle{remark}
\newtheorem*{note}{Note}
\newtheorem{case}{Case}

%this has equations numbered within sections 1.1,1.2, ... 2.1,...
\numberwithin{equation}{section}

\makeatletter
\newenvironment{solution}
               {\let\oldqedsymbol=\qedsymbol%
                \def\@addpunct##1{}%
                \renewcommand{\qedsymbol}{$\blacktriangleleft$}%
                \begin{proof}[\itshape Solution.]}%
               {\end{proof}%
                \renewcommand{\qedsymbol}{\oldqedsymbol}}
\makeatother

\makeatletter
\tikzset{join/.code=\tikzset{after node path={%
\ifx\tikzchainprevious\pgfutil@empty\else(\tikzchainprevious)%
edge[every join]#1(\tikzchaincurrent)\fi}}}
\makeatother

\tikzset{>=stealth',every on chain/.append style={join},
         every join/.style={->}}




\def\upint{\mathchoice%
    {\mkern13mu\overline{\vphantom{\intop}\mkern7mu}\mkern-20mu}%
    {\mkern7mu\overline{\vphantom{\intop}\mkern7mu}\mkern-14mu}%
    {\mkern7mu\overline{\vphantom{\intop}\mkern7mu}\mkern-14mu}%
    {\mkern7mu\overline{\vphantom{\intop}\mkern7mu}\mkern-14mu}%
  \int}
\def\lowint{\mkern3mu\underline{\vphantom{\intop}\mkern7mu}\mkern-10mu\int}


%-------------------------------------------
%       Begin Local Macros
%-------------------------------------------

\newcommand{\Z}{\mathbb{Z}}
\newcommand{\N}{\mathbb{N}}
\newcommand{\Q}{\mathbb{Q}}
\newcommand{\R}{\mathbb{R}}
\newcommand{\C}{\mathbb{C}}
\newcommand{\T}{\mathbb{T}}
\newcommand{\D}{\displaystyle}
\newcommand{\im}{\operatorname{im}}
\newcommand{\coker}{\operatorname{coker}}
\newcommand{\ind}{\operatorname{ind}}
\newcommand{\rank}{\operatorname{rank}}
\newcommand\mc[1]{\marginpar{\sloppy\protect\footnotesize #1}}

%-------------------------------------------
%       End Local Macros
%-------------------------------------------
\begin{document}
\title[MATH 697]{Introduction to Homological Algebra}

% \author{Robert Cardona}

\author{
	Robert Cardona %\textit{mrrobertcardona@gmail.com}
	\and
	Massy Khoshbin %\textit{massy255@gmail.com}
	\and
	Siavash Mortezavi %\textit{siavash.mortezavi@gmail.com}
}


\address{Department of Mathematics \\ California State University Long Beach}
\email{mrrobertcardona@gmail.com \and massy255@gmail.com \and siavash.mortezavi@gmail.com}

\date{\today}


\maketitle

%%%%%%%%%%%%%%%%%%%%%%%%%%%%%%%%%%%%%%%%%%%%%%%%
\setcounter{section}{-1}
\section{MATH 697 Homework Zero.Two}
%%%%%%%%%%%%%%%%%%%%%%%%%%%%%%%%%%%%%%%%%%%%%%%%


\textbf{ AM 2.1}: Show that $(\Z/m\Z) \otimes (\Z/n\Z) = 0$ if $m$ and $n$ are coprime.
	\begin{proof}
		Choose $a \otimes b \in \Z/m\Z \otimes \Z/n\Z$. Since $m$ and $n$ are coprime, there exist $s, t \in \Z$ such that $ms + nt = 1$ Observe that $$a = a \cdot 1 = a(ms + nt) = ams + ant \equiv ant \pmod m.$$ Now observe that $$a \otimes b = atn \otimes b = a \otimes nb = at \otimes 0 = 0.$$
		We have shown that any simple tensor is zero, so any finite linear combination of simple tensors is zero. Conclude $(\Z/m\Z) \otimes (\Z/n\Z) = 0$.\\
	\end{proof}

\textbf{ AM 2.2}: Let $R$ be a ring, $I$ an ideal of $R$, $M$ an $R$-module. Show that $(R/I) \otimes_RM$ is isomorphic to $M/IM$. 
	\begin{proof}
		Define $\varphi : R/I \times M \to M/IM$ by $\varphi(r + I, m) = rm + IM$, which we shall henceforth write as $\varphi(\overline{r}, m) = \overline{rm}$. Let $(\overline{r}, m) = (\overline{s}, m)$. Then $\overline{r} = \overline{s} \implies r \in \overline{s} \implies r=s+i$, some $i \in I$. Then $\varphi(\overline{r},m)=\overline{rm}=\overline{(s+i)m}=\overline{sm+im}=\overline{sm}+\overline{im}=\overline{sm}+\overline{0}=\overline{sm}=\varphi(\overline{s},m)$. Thus $\varphi$ is well-defined.\\ 
		
		Observe $\varphi(\overline{r}+\overline{s},m)=\varphi(\overline{r+s},m)=\overline{(r+s)m}=\overline{rm+sm}=\overline{rm}+\overline{sm}=\varphi(\overline{r},m)+\varphi(\overline{s},m)$. Similarly, $\varphi(\overline{r},m+n)=\varphi(\overline{r},m)+\varphi(\overline{r},n)$. Lastly, $\varphi(\overline{rs},m)=\overline{(rs)m}=\overline{r(sm)}=\varphi(\overline{r},sm)$. Thus $\varphi$ is $R$-biadditive (In fact, $\varphi$ is $R$-bilinear).\\
		
		 Now we are guaranteed a unique $R$-homomorphism $\phi : R/I \otimes_RM \rightarrow M/IM$ given by $\phi(\overline{r} \otimes m)=\overline{rm}$. Notice if we define $f : M/IM \rightarrow R/I \otimes_RM$ via $f(\overline{m})=\overline{1}\otimes m$ then $f$ is a $\Z$-homomorphism which makes $f \circ \phi$ and $\phi \circ f$ the identity map in $R/I \otimes_RM$ and $M/IM$, respectively. So $\phi$ has a two-sided inverse, hence a bijective function, and accordingly is an isomorphism when considered as an $R$-map. 
	\end{proof}
		 
\textbf{R 2.28}: Let $R$ be a domain with $Q$ = Frac($R$), its field of fractions. If $A$ is an $R$-module, prove that every element of $Q \otimes_RA$ has the form $q \otimes a$ for $q \in Q $ and $a \in A$ (i.e. every element is a simple tensor).
	\begin{proof}
		Let $\sum_1^n q_i \otimes a_i \in Q \otimes_RA$. We can write $\sum_1^n q_i \otimes a_i = \sum_1^n \frac{r_i}{s_i} \otimes a_i$ for $r_i,s_i \in R, s_i \neq 0$. Write $s=s_1s_2\cdots s_n$ and $\widehat{s_i}=\frac{s}{s_i}$. Then $\sum_1^n \frac{r_i}{s_i} \otimes a_i = \sum_1^n (1 \cdot \frac{r_i}{s_i}) \otimes a_i = \sum_1^n (\frac{\widehat{s_i}}{\widehat{s_i}} \cdot \frac{r_i}{s_i}) \otimes a_i = \sum_1^n \frac{\widehat{s_i}r_i}{s} \otimes a_i = \sum_1^n (\frac{1}{s})\widehat{s_i}r_i \otimes a_i = \sum_1^n \frac{1}{s} \otimes (\widehat{s_i}r_i)a_i = \frac{1}{s} \otimes (\sum_1^n \widehat{s_i}r_ia_i)$.\\
	\end{proof}

\textbf{R 2.32}: Consider the following commutative diagram in $_R$\textbf{Mod} having exact columns. \\

	
If the bottom two rows are exact, prove that the top row is exact; if the top two rows are exact, prove that the bottom row is exact. 
	\begin{proof}$\alpha_1$ is injective: Let $a' \in$ ker $\alpha_1$. Then $\alpha_1(a')=0$. So $f(\alpha_1(a'))=0$. Now $0=f(\alpha_1(a'))=\beta_1(f'(a'))$ by commutativity. The injectivity of $\beta_1$ implies $f'(a')=0$ and the injectivity of $f'$ gives us $a'=0$. Thus ker $\alpha_1=0$ and $\alpha_1$ is injective.\\
	
	$\alpha_2$ is surjective: \\
	
	im $\alpha_1\subseteq$ ker $\alpha_2$: Let $a \in$ im $\alpha_1$. Then there exists $a' \in A'$ with $a=\alpha_1(a')$. Observe $f(a)=f(\alpha_1(a'))=\beta_1(f'(a'))$ by commutativity. Thus $\beta_2(f(a))=\beta_2(\beta_1(f'(a')))=0$ by exactness. Now $0=\beta_2(f(a))=f''(\alpha_2(a))$ by commutativity. The injectivity of $f''$ gives us $\alpha_2(a)=0$. Hence $a \in$ ker $\alpha_2$.\\
	
	ker $\alpha_2 \subseteq$ im $\alpha_1$: Let $a  \in$ ker $\alpha_2$. Then $\alpha_2(a)=0$. So $f''(\alpha_2(a))=0$. By commutativity, $\beta_2(f(a))=0$. Now $f(a) \in$ ker $\beta_2 =$ im $\beta_1$, so there exists $b' \in B'$ such that $f(a)=\beta_1(b')$. Now $g(f(a))=0$ by exactness, so $g(\beta_1(b'))=0$. By commutativity, $\gamma_1(g'(b'))=0$. Since $\gamma_1$ is injective, $g'(b')=0$. Now $b' \in$ ker $g'=$ im $f'$ so there exists $a' \in A'$ such that $b'=f'(a')$. Thus $f(a)=\beta_1(b')=\beta_1(f'(a'))$. By commutativity, $\beta_1(f'(a'))=f(\alpha_1(a'))$. So $f(a)=f(\alpha_1(a'))$. Since $f$ is injective, we have $a=\alpha_1(a')$, and therefore $a\in$ im $\alpha_1$. \\
 \end{proof}






\end{document}
