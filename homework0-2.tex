\documentclass[8pt]{amsart}
\usepackage{amsmath, amssymb}
\usepackage{amsfonts}
\usepackage{mathrsfs}
\usepackage[arrow,matrix,curve,cmtip,ps]{xy}
\usepackage{paralist}
\usepackage[left=1in,top=1in,right=1in,bottom=1in]{geometry}
\usepackage{amsthm}

\allowdisplaybreaks

\theoremstyle{plain}% default

\theoremstyle{definition}
\newtheorem{theorem}{Theorem}[section]
\newtheorem{lemma}{Lemma}[section]
\newtheorem*{proposition}{Proposition}
\newtheorem*{corollary}{Corollary}
\newtheorem*{KL}{Klein’s Lemma}

\newtheorem*{definition}{Definition}
\newtheorem{conjecture}{Conjecture}[section]
\newtheorem{example}{Example}[section]
\newtheorem*{exercise}{Exercise}%[section]
\newtheorem*{notation}{Notation}
\newtheorem*{remark}{Remark}

\theoremstyle{remark}
\newtheorem*{note}{Note}
\newtheorem{case}{Case}

%this has equations numbered within sections 1.1,1.2, ... 2.1,...
\numberwithin{equation}{section}

\makeatletter
\newenvironment{solution}
               {\let\oldqedsymbol=\qedsymbol%
                \def\@addpunct##1{}%
                \renewcommand{\qedsymbol}{$\blacktriangleleft$}%
                \begin{proof}[\itshape Solution.]}%
               {\end{proof}%
                \renewcommand{\qedsymbol}{\oldqedsymbol}}
\makeatother

\def\upint{\mathchoice%
    {\mkern13mu\overline{\vphantom{\intop}\mkern7mu}\mkern-20mu}%
    {\mkern7mu\overline{\vphantom{\intop}\mkern7mu}\mkern-14mu}%
    {\mkern7mu\overline{\vphantom{\intop}\mkern7mu}\mkern-14mu}%
    {\mkern7mu\overline{\vphantom{\intop}\mkern7mu}\mkern-14mu}%
  \int}
\def\lowint{\mkern3mu\underline{\vphantom{\intop}\mkern7mu}\mkern-10mu\int}


%-------------------------------------------
%       Begin Local Macros
%-------------------------------------------

\newcommand{\Z}{\mathbb{Z}}
\newcommand{\N}{\mathbb{N}}
\newcommand{\Q}{\mathbb{Q}}
\newcommand{\R}{\mathbb{R}}
\newcommand{\C}{\mathbb{C}}
\newcommand{\T}{\mathbb{T}}
\newcommand{\D}{\displaystyle}
\newcommand{\im}{\operatorname{im}}
\newcommand{\coker}{\operatorname{coker}}
\newcommand{\ind}{\operatorname{ind}}
\newcommand{\rank}{\operatorname{rank}}
\newcommand\mc[1]{\marginpar{\sloppy\protect\footnotesize #1}}

%-------------------------------------------
%       End Local Macros
%-------------------------------------------
\begin{document}
\title[MATH 697]{Introduction to Homological Algebra}

% \author{Robert Cardona}

\author{
	Robert Cardona %\textit{mrrobertcardona@gmail.com}
	\and
	Massy Khoshbin %\textit{massy255@gmail.com}
	\and
	Siavash Mortezavi %\textit{siavash.mortezavi@gmail.com}
}


\address{Department of Mathematics \\ California State University Long Beach}
\email{mrrobertcardona@gmail.com \and massy255@gmail.com \and siavash.mortezavi@gmail.com}

\date{\today}


\maketitle

%%%%%%%%%%%%%%%%%%%%%%%%%%%%%%%%%%%%%%%%%%%%%%%%
\setcounter{section}{-1}
\section{MATH 697 Homework Zero.Two}
%%%%%%%%%%%%%%%%%%%%%%%%%%%%%%%%%%%%%%%%%%%%%%%%


\textbf{ AM 2.1}: Show that $(\Z/m\Z) \otimes (\Z/n\Z) = 0$ if $m$ and $n$ are coprime.
	\begin{proof}
		Observe that since $m$ and $n$ are relatively prime, there exists $s, t \in \Z$ such that $ms + nt = 1$. Now let $a \otimes b \in (\Z/m\Z) \otimes (\Z/n\Z)$. Observe that
		\begin{align*}
			a \otimes b &= a \cdot 1 \otimes b \cdot 1\\
			&= a (ms + nt) \otimes b (ms + nt)\\
			&= (ams + ant) \otimes (bms + bnt)\\
			&= ams \otimes (bms + bnt) + ant \otimes (bms + bnt)\\
			&= ams \otimes bms + ams \otimes bnt + ant \otimes bms + ant \otimes bnt\\
			&= 0 \otimes bms + ams \otimes 0 + ant \otimes bms + ant \otimes 0\\
			&= ant \otimes bms\\
			&= atn \otimes bms\\
			&= at \otimes nbms\\
			&= at \otimes 0\\
			&= 0
		\end{align*}
		Hence we have shown that any simple tensor is zero, so any finite linear combination of simple tensors is zero.\\

		Conclude that  $(\Z/m\Z) \otimes (\Z/n\Z) = 0$ holds.
	\end{proof}

\textbf{ AM 2.2}: Let $A$ be a ring, $a$ an ideal, $M$ an $A$-module. Show that $(A/a) \otimes_AM$ is isomorphic to $M/aM$. [Tensor the exact sequence $0 \longrightarrow a \longrightarrow A \longrightarrow A/a \longrightarrow 0$ with $M$.]
	\begin{proof}
	
	\end{proof}

\end{document}




